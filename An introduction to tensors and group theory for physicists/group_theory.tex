\documentclass[10pt]{article}
\pdfoutput=1
%\usepackage{NotesTeX,lipsum}
\usepackage{NotesTeX,lipsum}
%\usepackage{showframe}

\title{\begin{center}{\Huge \textit{Notes}}\\{{\itshape Tensors and Group theory}}\end{center}}
\author{Yi Huang}


\affiliation{
University of Minnesota
}

\emailAdd{yihphysics@gmail.com}

\begin{document}
	\maketitle
	\flushbottom
	\newpage
	\pagestyle{fancynotes}
	\part{Problems}
	\section{Chapter 2}\label{chp: 2}
	\begin{margintable}\vspace{.8in}\footnotesize
		\begin{tabularx}{\marginparwidth}{|X}
		Section~\ref{chp: 2}. Chapter 2\\
		\end{tabularx}
	\end{margintable}

	This note is based on \textit{An Introduction to Tensors and Group Theory for Physicsts} by Nadir Jeevanjee.

\subsection{2.1}
Prove that $L^2([-a,a])$ is closed under addition. You will need the triangle inequlity, as well as the following inequality, valid for all $\lambda \in \mathbb{R}: 0 \le \int_{-a}^a (\abs{f} + \abs{g})^2 \dd{x}$.

\begin{proof}
	Suppose $f,g \in L^2([-a,a])$, let $\lambda = -1$, and we have
	\begin{equation}
		2\int_{-a}^a \abs{f}\abs{g} \dd{x} \le \int_{-a}^a \abs{f}^2 \dd{x} + \int_{-a}^a \abs{g}^2 \dd{x} < \infty,
	\end{equation}
	so
	\begin{equation}
		\int_{-a}^a (\abs{f} + \abs{g})^2 \dd{x} = \int_{-a}^a (\abs{f}^2 + \abs{g}^2 + 2\abs{f}\abs{g}) \dd{x} < \infty.
	\end{equation}
	Using the triangle inequality, we have
	\begin{equation}
		\int_{-a}^a \abs{f+g}^2 \dd{x} \le \int_{-a}^a (\abs{f} + \abs{g})^2 \dd{x} < \infty.
	\end{equation}
	So $L^2([-a,a])$ is closed under addition.
\end{proof}

\subsection{2.2}
In this problem we show that $\{r^l Y_m^l\}$ is a basis for $\mathcal{H}_l(\mathbb{R}^3)$, which implies that $\{Y_m^l\}$ is a basis for $\tilde{\mathcal{H}}_l(\mathbb{R}^3)$.
\begin{enumerate}[(a)]
	\item Let $f \in \mathcal{H}_l(\mathbb{R}^3)$, and write $f$ as $f = r^l Y(\theta, \phi)$.
	Then we know that
	\begin{equation}
		\Delta_{S^2} Y = -l(l+1) Y, \label{eq: 1}
	\end{equation}
	where
	\begin{equation}
		\Delta_{S^2} = \pdv[2]{\theta} + \cot{\theta} \pdv{\theta} + \frac{1}{\sin^2{\theta}}\pdv[2]{\phi}.
	\end{equation}
	If you have never done so, use the expression for $\Delta_{S^2}$ and the expression for the angular momentum operators to show that
	\begin{equation}
		-\Delta_{S^2} = L_x^2 + L_y^2 + L_z^2 \equiv \vb{L}^2,
	\end{equation}
	where
	\begin{gather*}
		L_x = -i\qty(y \pdv{z} - z \pdv{y}), \\
		L_y = -i\qty(z \pdv{x} - x \pdv{z}), \\
		L_z = -i\qty(x \pdv{y} - y \pdv{x}). \\
	\end{gather*}
	so that \eqref{eq: 1} says that $Y$ is an eigenfuction of $\vb{L}^2$, as expected. You will need to convert between cartesian and spherical coordinates. The theory of angular momentum then tells us that $\mathcal{H}_l(\mathbb{R}^3)$ has dimension $2l+1$.

	\begin{proof}
		We need to express the cartesian coordinates $\{x,y,z\}$ and the corresponding derivatives $\{\partial_x, \partial_y, \partial_z\}$ in terms of spherical coordinates
		\begin{gather*}
			x = r \sin{\theta} \cos{\phi}\qc y = r \sin{\theta}\sin{\phi}\qc z = r\cos{\theta}, \\
			r = \sqrt{x^2 + y^2 + z^2}\qc \theta = \arccos{\frac{z}{r}}\qc \phi = \arctan{\frac{y}{x}}.
		\end{gather*}
		Hence
		\begin{align*}
			\dd{f} &= \pdv{f}{x} \dd{x} + \pdv{f}{y} \dd{y} + \pdv{f}{z} \dd{z} \\
			&= \pdv{f}{r} \dd{r} + \pdv{f}{\theta} \dd{\theta} + \pdv{f}{\phi} \dd{\phi} \\
			&= \qty(\pdv{f}{r})_{\theta, \phi} \qty(\pdv{r}{x} \dd{x} + \pdv{r}{y} \dd{y} + \pdv{r}{z} \dd{z}) \\
			&+ \qty(\pdv{f}{\theta})_{r, \phi} \qty(\pdv{\theta}{x} \dd{x} + \pdv{\theta}{y} \dd{y} + \pdv{\theta}{z} \dd{z}) \\
			&+ \qty(\pdv{f}{\phi})_{r,\theta} \qty(\pdv{\phi}{x} \dd{x} + \pdv{\phi}{y} \dd{y} + \pdv{\phi}{z} \dd{z})
		\end{align*}
		where we need to take care of what variables are kept constant when we take the partial derivatives, because we have two sets of variables $(x,y,z)$ and $(r,\theta, \phi)$.
		It is not difficult to show
		\begin{gather*}
			\qty(\pdv{r}{x})_{y,z} = \frac{x}{r}\qc \qty(\pdv{r}{y})_{x,z} = \frac{y}{r}\qc \qty(\pdv{r}{z})_{x,y} = \frac{z}{r}, \\
			\qty(\pdv{\theta}{x})_{y,z} = \frac{\cos{\theta}\cos{\phi}}{r}\qc
			\qty(\pdv{\theta}{y})_{x,z} = \frac{\cos{\theta}\sin{\phi}}{r}\qc
			\qty(\pdv{\theta}{z})_{x,y} = -\frac{\sin{\theta}}{r}, \\
			\qty(\pdv{\phi}{x})_{y,z} = -\frac{sin{\phi}}{r\sin{\theta}}\qc \qty(\pdv{\phi}{y})_{x,z} = \frac{\cos{\phi}}{r\sin{\theta}}. \\
		\end{gather*}
		Therefore we have
		\begin{align*}
			\pdv{f}{x} &= \qty(\pdv{r}{x})_{y,z} \qty(\pdv{f}{r})_{\theta, \phi} +  \qty(\pdv{\theta}{x})_{y,z} \qty(\pdv{f}{\theta})_{r, \phi} +
			\qty(\pdv{\phi}{x})_{y,z} \qty(\pdv{f}{\phi})_{r,\theta} \\
			&= \qty(\sin{\theta}\cos{\phi} \pdv{r} + \frac{\cos{\theta}\cos{\phi}}{r} \pdv{\theta} - \frac{\sin{\phi}}{r\sin{\theta}} \pdv{\phi})f,\\
			\pdv{f}{y} &= \qty(\pdv{r}{y})_{x,z} \qty(\pdv{f}{r})_{\theta, \phi} +  \qty(\pdv{\theta}{y})_{x,z} \qty(\pdv{f}{\theta})_{r, \phi} +
			\qty(\pdv{\phi}{y})_{x,z} \qty(\pdv{f}{\phi})_{r,\theta} \\
			&= \qty(\sin{\theta}\sin{\phi} \pdv{r} + \frac{\cos{\theta}\sin{\phi}}{r} \pdv{\theta} + \frac{\cos{\phi}}{r\sin{\theta}} \pdv{\phi})f,\\
			\pdv{f}{z} &= \qty(\pdv{r}{z})_{x,y} \qty(\pdv{f}{r})_{\theta, \phi} +  \qty(\pdv{\theta}{z})_{x,y} \qty(\pdv{f}{\theta})_{r, \phi} +
			\qty(\pdv{\phi}{z})_{x,y} \qty(\pdv{f}{\phi})_{r,\theta} \\
			&= \qty(\cos{\theta} \pdv{r} - \frac{\sin{\theta}}{r} \pdv{\theta})f, \\
		\end{align*}
		Now we can calculate the angular momentum operators in spherical coordinates
		\begin{align*}
			L_x &= -i\qty(y \pdv{z} - z \pdv{y}) = i \qty(\sin{\phi} \pdv{\theta} + \cot{\theta}\cos{\phi}\pdv{\phi}), \\
			L_y &= -i\qty(z \pdv{x} - x \pdv{z}) = -i \qty(\cos{\phi} \pdv{\theta} - \cot{\theta}\sin{\phi}\pdv{\phi}) \\
			L_z &= -i\qty(x \pdv{y} - y \pdv{x}) = -i \pdv{\phi}, \\
		\end{align*}
		and we define the ladder operators $L_{\pm} = L_x \pm i L_y$, so we have
		\begin{equation}
			L_{\pm} = \e^{\pm i \phi} \qty(\pm \pdv{\theta} + i \cot{\theta} \pdv{\phi}).
		\end{equation}
		and
		\begin{align*}
			\vb{L}^2 &\equiv L_x^2 + L_y^2 + L_z^2 \\
			&= L_+L_- + i[L_x,L_y] + L_z^2 \\
			&= L_+L_- + L_z^2 - L_z \\
			&= -\pdv[2]{\theta} - \cot{\theta} \pdv{\theta} - \frac{1}{\sin^2{\theta}}\pdv[2]{\phi}.
		\end{align*}
		where we use $L_z = i[L_x, L_y]$.
	\end{proof}
	\item Exhibit a basis for $\mathcal{H}_l(\mathbb{R}^3)$ by considering the function $f_0^l \equiv (x+iy)^l$ and showing that
	\begin{equation}
		L_z(f_0^l) = l f_0^l\qc L_+(f_0^l) \equiv (L_x + iL_y) (f_0^l).
	\end{equation}
	The theory of angular momentum then tells us that $(L_-)^k f_0^l \equiv f_k^l$ satisfies $L_z f_k^l = (l-k) f_k^l$ and that $\{f_k^l\}_{0\le k \le 2l}$ is a basis for $\mathcal{H}_l(\mathbb{R}^3)$.
	\begin{proof}
		Rewrite $f_0^l$ in spherical coordinates
		\begin{equation}
			f_0^l = r^l \sin^l{\theta} \e^{il \phi}.
		\end{equation}
		Hence
		\begin{equation}
			L_z f_0^l = -i \pdv{f_0^l}{\phi} = l f_0^l.
		\end{equation}
		Apply ladder operators to $f_0^l$, we have
		\begin{align*}
			L_+(f_0^l) &= \e^{i \phi} \qty(\pdv{\theta} + i \cot{\theta} \pdv{\phi}) \qty(r^l \sin^l{\theta} \e^{il \phi}) \\
			&= r^l \e^{i \phi} \e^{il \phi} \qty(l (\sin{\theta})^{l-1} \cos{\theta} - l \cot{\theta} \sin^l{\theta}) = 0.
		\end{align*}
	\end{proof}
\end{enumerate}

\subsection{Exercise 2.6}
Suppose $V$ is finite-dimensional and let $T \in \mathcal{L}(V)$. Show that $T$ being one-to-one is equivalent to $T$ being onto. Feel free to introduce a basis to assist you in the proof.

\begin{proof}
	Choose a basis $\mathcal{B} = \{e_1,e_2, \dots, e_n\}$ of $V$, and $\dim{V} = n$.

	\begin{enumerate}
		\item Given $T$ is one-to-one, we can prove that $\mathcal{B'} = \{T (e_i)\}$ is also a basis of $V$, i.e., $\mathcal{B'}$ is independent and spans $V$.

		We write the linear relation for $T(e_i)$
		\begin{equation}
			\sum_{i=1}^{n} c_i T(e_i) = 0.
		\end{equation}
		Because $T$ is linear, we can also write
		\begin{equation}
			T(c_j e_j) = T\qty(-\sum_{i\neq j} c_i e_i).
		\end{equation}
		Given $T$ is injective, we have
		\begin{equation}
			c_j e_j = -\sum_{i\neq j} c_i e_i, \qor \sum_{i=1}^n c_i e_i = 0.
		\end{equation}
		Then $c_i = 0$ for $i=1,2,\dots,n$, because $\mathcal{B}$ is linear independent. Hence
		$\mathcal{B'}$ is linear independent. Since the nember of elements of $\mathcal{B'}$ is $n$, $\mathcal{B'}$ is also a basis of $V$.

		For any vector $w \in V$, we can expand it in basis $\mathcal{B'}$, such that
		\begin{equation}
			w = \sum_{i=1}^n a_i T(e_i) = T \qty(\sum_{i=1}^n a_i e_i) = T(v),
		\end{equation}
		where $v = \sum_i a_i e_i \in V$. Hence $\exists v\in V$, such that $w = T(v)$, so $T$ is onto.

		\item Given $T$ is onto, for any $w \in V$, there exists $v$ such that $w = T(v)$. If we want to show $T$ is also one-to-one, then we need to prove such $v$ is unique. We use proof by contradictions. Suppose $\exists v_1 \neq v_2$, such that $w = T(v_1) = T(v_2)$. Because $T$ is linear, we have
		\begin{equation}
			T(v_1) - T(v_2) = T(v_1 - v_2) = 0. \label{eq: e2.6.1}
		\end{equation}
		Write $v_{1,2}$ in basis $\mathcal{B}$
		\begin{gather*}
			v_1 = \sum_{i=1}^n c_{1i} e_i\qc v_2 = \sum_{i=1}^n c_{2i} e_i, \\
			v_1 - v_2 = \sum_{i=1}^n c_{i} e_i \neq 0.
		\end{gather*}
		where $c_i = c_{1i} - c_{2i}$, and there exists some $j \in \{1,2,\dots,n\}$ such that $c_j \neq 0$ because $v_1 - v_2 \neq 0$. Hence \eqref{eq: e2.6.1} can be written as
		\begin{equation}
			\sum_{i=1}^n c_i T(e_i) = 0\qc \exists j \in \{1,2,\dots,n\} \text{such that} c_j \neq 0,
		\end{equation}
		which means $\mathcal{B'}=\{T(e_i)\}$ is linearly dependent, $\dim(\mathcal{B'}) < n$, so there exists $w' \notin V$, with $w'$ can not be express in terms of a linear combination of basis $\mathcal{B'}$. In other words, we cannot find such $v' = \sum_i a_i e_i \in V$ such that $w' = \sum_i a_i T(e_i) = T(v')$. This contradicts our assumption that $T$ is onto. Therefore $v_1 = v_2 = v$ is unique once we are given a $w \in V$. In other words
		\begin{equation}
			w = T(v_1) = T(v_2) \Longrightarrow v_1 = v_2, \label{eq: e2.6.2}
		\end{equation}
		if $w \in \mathrm{Span}(\mathcal{B'})$, then $v \in \mathrm{Span}(\mathcal{B})$, i.e. \eqref{eq: e2.6.2} is true for all $v_1 = v_2 \in V$. Eventually we prove that $T$ is one-to-one.
	\end{enumerate}
\end{proof}

\subsection{Exercise 2.7}
Suppose $T(v) = 0 \Longrightarrow v = 0$. Show that this is equivalent to $T$ being one-to-one.

\begin{proof}
	Write $v = v_1 - v_2$, because $T$ is linear
	\begin{equation}
		T(v) = T(v_1 - v_2) = T(v_1) - T(v_2) = 0 \Longrightarrow v = v_1 - v_2 =0.
	\end{equation}
	Hence $T$ is one-to-one.
\end{proof}

\subsection{Exercise 2.10}
By carefully working with the definitions, show that the $e^i$ defined in (2.18) and satisfying (2.20) are linearly independent.

\begin{proof}
	For any vector $v \in V$ such that
	\begin{equation}
		v = \sum_{i=1}^n v^i e_i,
	\end{equation}
	a dual vector $e^i$ is defined by
	\begin{equation}
		e^i (v) \equiv v^i.
	\end{equation}
	If $v = e_j = \sum_i \delta_j^i e_i$, then $e^i(e_j) = \delta_j^i$.
	We are well-prepared to prove $\mathcal{B}^* = \{e^i\}$ is linearly independent.

	First we write a linear relation for $\{e^i\}$
	\begin{equation}
		\sum_{i=1}^n c_i e^i = 0 \label{eq: e2.10.1}
	\end{equation}
	where $c_i \in C$ is a scalar. Remember that $e^i$ is a $C$-valued function on $V$, so if we apply this linear relation \eqref{eq: e2.10.1} on any vector $v \in V$, we must still have zero. Let $v = e_j$, where $j = 1, 2, \dots, n$, then
	\begin{equation}
		\sum_{i=1}^n c_i e^i (e_j) = \sum_{i=1}^n c_i \delta_j^i = c_j = 0.
	\end{equation}
	In other words the linear relation \eqref{eq: e2.10.1} is true, unless all $c_i = 0$. Therefore $\mathcal{B}^*$ is linear independent.
\end{proof}

\subsection{Excercise 2.11}
Let $(\cdot|\cdot)$ be an inner product. If a set of non-zero vectors $e_1, \dots, e_k$ is orthogonal, i.e. $(e_i|e_j) = 0$ when $i \neq j$, show that they are linearly independent. Note that an orthonormal set (i.e. $(e_i|e_j) = \pm \delta_{ij}$) is just an orthogonal set in which the vectors have unit length.

\begin{proof}
	Similar to the previous exercise, we write a linear relation for $e_1, \dots, e_k$
	\begin{equation}
		\sum_{j=1}^k c^j e_j = 0.
	\end{equation}
	We have the following inner product between $e_i$ ($i = 1, 2, \dots, k$) and the left hand side of the linear relation
	\begin{equation}
		\qty(e_i\Bigg{|}\sum_{j=1}^k c^j e_j) = c^i (e_i|e_i) = 0.
	\end{equation}
	Notice that $(e_i|e_i) > 0$ since $(\cdot|\cdot)$ is an inner product, so we must have $c^i = 0$ where $i = 1, 2, \dots, k$. Hence $\{e_1, \dots, e_k\}$ is linear independent.
\end{proof}

\subsection{Exercise 2.12}
Let $A,B \in M_n(\mathbb{C})$. Define $(\cdot|\cdot)$ on $M_n(\mathbb{C})$ by
\begin{equation}
	(A|B) = \frac{1}{2} \Tr(A^{\dagger}B).
\end{equation}
Check that this is indeed an inner product. Also check that the basis $\{I, \sigma_x, \sigma_y, \sigma_z\}$ for $H_2(\mathbb{C})$ is orthonormal with respect to this inner product.

\begin{proof}
	Condition 1, linearity in the second argument
	\begin{align*}
		(A|cB) &= \frac{1}{2} \Tr(A^{\dagger}cB) \\
		&= c \frac{1}{2} \Tr(A^{\dagger}B) \\
		&= c(A|B).
	\end{align*}
	Condition 2, Hermiticity
	\begin{align*}
		\overline{(B|A)} &= \frac{1}{2}  \Tr[(B^{\dagger}A)^{\dagger}] \\
		&= \frac{1}{2} \Tr(A^{\dagger}B) \\
		&= (A|B).
	\end{align*}
	Condition 4, positive-definiteness
	\begin{equation}
		(A|A) = \frac{1}{2} \Tr(A^{\dagger}A) = \frac{1}{2} \sum_{i,j} \abs{a_{ij}}^2 > 0\qif A \neq 0.
	\end{equation}
	Condition 4 implies condition 3 (non-degeneracy), so this is indeed a inner product.

	As for the basis of $H_2(\mathbb{C})$, recall the property of Pauli matrices
	\begin{equation}
		\sigma_i \sigma_j = i \epsilon_{ijk} \sigma_k + \delta_{ij},
	\end{equation}
	which can be derived from
	\begin{equation}
		[\sigma_i, \sigma_j]_- = 2i \epsilon_{ijk} \sigma_k\qc [\sigma_i, \sigma_j]_+ = 2 \delta_{ij}.
	\end{equation}
	Hence
	\begin{equation}
		(\sigma_i|\sigma_j) = \frac{1}{2} \Tr(\sigma_i \sigma_j) = \delta_{ij}.
	\end{equation}
	And it is easy to show
	\begin{equation}
		(I|\sigma_j) = 0\qc (I, I) = 1.
	\end{equation}
	Therefore the basis $\{I, \sigma_x, \sigma_y, \sigma_z\}$ for $H_2(\mathbb{C})$ is orthonormal with respect to this inner product.
\end{proof}

\subsection{Exercise 2.13}
Let $v = (x,y,z,t)$ be an arbitrary non-zero vector in $\mathbb{R}^4$. Show that $\eta$ is non-degenerate by finding another vector $w$ such that $\eta(v, w) \neq 0$.

\begin{proof}
	Since $v = (x,y,z,t)$ is non-zero, there exists at least one non-zero component in $v$. Without losing generality, say this non-zero component is space component $x$. Let $w = (x, 0, 0, 0)$, so we have
	\begin{equation}
		\eta(v,w) = x^2 > 0,
	\end{equation}
	and it is easy to see that if the non-zero component is time, then we can set $w = (0,0,0,t)$ such that $\eta(v,w) = -t^2 < 0$. In either way we find another vector $w$ such that $\eta(v,w) \neq 0$.

	We can also prove its contrapositive proposition. If $\eta(v,w) = 0$ for any $w \in \mathbb{R}^4$, then $v = 0$.

	Suppose $w = (1, 0, 0, 0)$, then $\eta(v,w) = x =0$. Similarly we can prove all components of $v$ are zero.
\end{proof}

\subsection{Exercise 2.16}
Use the non-degeneracy of $(\cdot|\cdot)$ to show that $L$ is one-to-one, i.e. that $L(v) = L(w) \Longrightarrow v = w$. Combine this with the argument used in Exercise 2.7 to show that $L$ is onto as well.
\begin{proof}
	If $L(v) = L(w)$, then
	\begin{equation}
		L(u) = L(v-w) = L(v) - L(w) = 0,
	\end{equation}
	where $u = v-w$. This is because $L$ is conjugate linear. According to the definition of metric dual $L$, for any vector $x \in V$, we have
	\begin{equation}
		L(u) x = (u|x) = 0 \label{eq: 2.16.1}
	\end{equation}
	because $L(u) = 0$. Then by the non-degeneracy of $(\cdot|\cdot)$, \eqref{eq: 2.16.1} implies $u=0$. Hence we prove $v = w$ if $L(v) = L(w)$, which means $L$ is one-to-one.

	We can employ similar argument shown in Excercise 2.6 and 2.7 to prove $L$ is invertible, because $V$ and $V^*$ have the same dimensions and the metric dual of all basis vector $e_i \in V$ also forms a basis $\{L(e_i)\} \subset V^*$.
\end{proof}

\subsection{Inner product with a zero vector must be zero}
Suppose we have an inner product $(\cdot|\cdot)$ defined in a vector space $V$, then the linearity implies the inner product with zero vector is zero. Explicitly if $c \in C$ is an arbitrary non-zero scalar in field $C$, we have
\begin{equation}
	(v|0) = (v|c 0) = c(v|0). \label{eq: 322.1}
\end{equation}
Suppose $(v|0) \neq 0$, then \eqref{eq: 322.1} means $c=1$ which contradicts our assumption that $c$ is an arbitrary scalar. Hence $(v|0) = 0$.

Actually this result is general for all linear or conjugate linear forms. 




\end{document}
