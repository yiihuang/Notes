\documentclass[10pt]{article}
\pdfoutput=1
%\usepackage{NotesTeX,lipsum}
\usepackage{NotesTeX,lipsum}
%\usepackage{showframe}

\title{\begin{center}{\Huge \textit{Notes}}\\{{\itshape Quantum Mechanics}}\end{center}}
\author{Yi Huang\footnote{\href{https://github.com/yiihuang}{\textit{My Github}}}}


\affiliation{
University of Minnesota
}

\emailAdd{yihphysics@gmail.com}

\begin{document}
	\maketitle
	\flushbottom
	\newpage
	\pagestyle{fancynotes}
	\part{Caprice}
	\section{Fall 2018}\label{sec:fall2018}
	\begin{margintable}\vspace{.8in}\footnotesize
		\begin{tabularx}{\marginparwidth}{|X}
		Section~\ref{sec:fall2018}. Fall 2018\\
		\end{tabularx}
	\end{margintable}

\subsection{Derivation of \rom{1}.2 (28) in Appendix 3 (Nutshell)}

To do an integral of form $I = \int_{-\infty}^{\infty} dq \exp[-f(x)/\hbar]$, we often have to resort to steepest-descent method. In limit of $\hbar \to 0$, the integral is dominated by the minimum of $f(a) = \min[f(q)]$, Expanding $f(q)$ around $a$
\begin{equation}
	f(q) = f(a)	+ \thalf f''(a) (q-a)^2 + \mcO[(q-a)^3]
\end{equation}
This is a Gaussian integral
\begin{align*}
	I &= \int_{-\infty}^{\infty} \dd q \, \e^{-f(x)/\hbar} \\
	&= \e^{-f(a)/\hbar} \left[\frac{2\pi \hbar}{f''(a)}\right]^{\thalf} \e^{-\mcO(\hbar^{\thalf})}
\end{align*}
Suppose now $f(\bfq)$ is a function of multiple variables $\bfq = (q_1, \dots, q_N)$, then the expandsion of $f(\bfq)$ around the equilibrium position $\bfa = (a_1, \dots, a_N)$ can be written as
\begin{equation}
	f(\bfq) = f(\bfa) + \thalf (\bfq - \bfa)^\top f''(\bfa) (\bfq - \bfa) + \mcO[|\bfq - \bfa|^3]
\end{equation}
where $f''(\bfq)$ is understood as the Hessian of $f(\bfq)$
\begin{equation}
	[f''(\bfq)]_{i,j} = \frac{\partial f}{\partial q_i \partial q_j}
\end{equation}
Therefore we rewrite the integral
\begin{align*}
	I &= \int \dd \bfq \, \e^{-f(\bfq/\hbar)}\\
	&= \e^{-f(\bfa)/\hbar} \int \dd \bfq \exp\left[ \tfrac{1}{2\hbar} (\bfq - \bfa)^\top f''(\bfa) (\bfq - \bfa) \right] \e^{-\mcO(\hbar^{\thalf})}
\end{align*}
where
\begin{align*}
	\lefteqn{\exp \left[ \tfrac{1}{2\hbar} (\bfq - \bfa)^\top f''(\bfa) (\bfq - \bfa) \right]} \\
	&= (\bfq - \bfa)^\top \sum_{k=0}^{\infty} \frac{[f''(\bfa)]^k}{k! (2\hbar)^k} (\bfq - \bfa)\\
	&= (\bfq - \bfa)^\top U^\top \sum_{k=0}^{\infty} \frac{[U f''(\bfa) U^\top]^k}{k! (2\hbar)^k} U(\bfq - \bfa)\\
	&= \bfy^\top \exp [U f''(\bfa) U^\top/2\hbar] \bfy
\end{align*}
where $U$ is an orthogonal matrix such that $U^\top U = \bbI$, and $\bfy = U \bfq$. Choose $U$ such that $Uf''(\bfa)U^\top$ is a diagonal matrix, with the diagonal elements equal to its eigenvalues
\begin{equation}
	\left[Uf''(\bfa)U^\top\right]_{i,i} = \mu_i
\end{equation}
Then
\begin{equation}
	\bfy^\top \exp\left[Uf''(\bfa)U^\top/2\hbar\right] \bfy = \exp\left(\tfrac{1}{2\hbar} \sum_i \mu_i y_i^2\right)
\end{equation}
Under the change of variables, the differential elements $\dd \bfq \to \dd \bfy = |\tfrac{\partial \bfy}{\partial \bfq}| \dd \bfq$, so the Jacobian is $\det(U) = 1$. Eventually we get the  product of Gaussian integrals
\begin{align*}
	I &= \int \dd \bfq \, \e^{-f(\bfq/\hbar)}\\
	&= \int \dd \bfy \exp\left(\tfrac{1}{2\hbar} \sum_i \mu_i y_i^2\right) \\
	&= \prod_{i=1}^N \int_{-\infty}^{\infty} \dd y_i \left[ \exp\left(\tfrac{1}{2\hbar} \mu_i y_i^2\right) \right] \\
	&= \left[ \frac{(2\pi \hbar)^N}{\prod_i \mu_i} \right]^{\thalf} = \left[ \frac{(2\pi \hbar)^N}{\det[f''(\bfa)]} \right]^{\thalf}
\end{align*}

\subsection{Contraction identities}

Show the following contraction identities are true:
\begin{gather*}
	\gamma_{\lambda} \gamma^{\lambda} = 4\qc \gamma_{\lambda} \gamma^{\alpha} \gamma^{\lambda} = -2 \gamma^{\alpha} \\
	\gamma_{\lambda} \gamma^{\alpha} \gamma^{\beta} \gamma^{\lambda} = 4 g^{\alpha \beta}\qc \gamma_{\lambda} \gamma^{\alpha} \gamma^{\beta} \gamma^{\gamma} \gamma^{\lambda} = -2 \gamma^{\gamma} \gamma^{\beta} \gamma^{\alpha} \\
	\gamma_{\lambda} \gamma^{\alpha} \gamma^{\beta} \gamma^{\gamma} \gamma^{\delta} \gamma^{\lambda} = 2(\gamma^{\delta} \gamma^{\alpha} \gamma^{\beta} \gamma^{\gamma} + \gamma^{\gamma} \gamma^{\beta} \gamma^{\alpha} \gamma^{\delta})
\end{gather*}

\begin{proof}
	\begin{align*}
		\gamma_{\lambda} \gamma^{\lambda} &= \half g_{\lambda \mu} \gamma^{\mu} \gamma^{\lambda} + \half g_{\mu \lambda} \gamma^{\lambda} \gamma^{\mu} \\
		&= \half g_{\mu \lambda} [\gamma^{\mu}, \gamma^{\lambda}]_+ \\
		&= g_{\mu \lambda} g^{\mu \lambda} = 4,
	\end{align*}
	where we use $g_{\lambda \mu} = g_{\mu \lambda}$ is a symmetric tensor.

	\begin{align*}
		\gamma_{\lambda} \gamma^{\alpha} \gamma^{\lambda} &= \gamma_{\lambda} (- \gamma^{\lambda} \gamma^{\alpha} + 2g^{\alpha \lambda}) \\
		&= -4 \gamma^{\alpha} + 2 \gamma^{\alpha} = -2 \gamma^{\alpha}.
	\end{align*}

	\begin{align*}
		\gamma_{\lambda} \gamma^{\alpha} \gamma^{\beta} \gamma^{\lambda} &= \gamma_{\lambda} \gamma^{\alpha} (-\gamma^{\lambda} \gamma^{\beta} + 2g^{\beta \lambda}) \\
		&= 2[\gamma^{\alpha} \gamma^{\beta}]_+ = 4g^{\alpha \beta}.
	\end{align*}

	\begin{align*}
		\gamma_{\lambda} \gamma^{\alpha} \gamma^{\beta} \gamma^{\gamma} \gamma^{\lambda} &= \gamma_{\lambda} \gamma^{\alpha} \gamma^{\beta} (-\gamma^{\lambda} \gamma^{\gamma} + 2g^{\gamma \lambda})\\
		&= -4 \gamma^{\gamma} g^{\alpha \beta} + 2 \gamma^{\gamma} \gamma^{\alpha} \gamma^{\beta}\\
		&= -2 (\gamma^{\gamma} \gamma^{\alpha} \gamma^{\beta} + \gamma^{\gamma} \gamma^{\beta} \gamma^{\alpha}) + 2 \gamma^{\gamma} \gamma^{\alpha} \gamma^{\beta}\\
		&= -2 \gamma^{\gamma} \gamma^{\beta} \gamma^{\alpha}.
	\end{align*}

	\begin{align*}
		\gamma_{\lambda} \gamma^{\alpha} \gamma^{\beta} \gamma^{\gamma} \gamma^{\delta} \gamma^{\lambda} &= \gamma_{\lambda} \gamma^{\alpha} \gamma^{\beta} \gamma^{\gamma} (-\gamma^{\lambda} \gamma^{\delta} + 2g^{\delta \lambda})\\
		&= 2(\gamma^{\delta} \gamma^{\alpha} \gamma^{\beta} \gamma^{\gamma} + \gamma^{\gamma} \gamma^{\beta} \gamma^{\alpha} \gamma^{\delta}).
	\end{align*}
\end{proof}

\end{document}
