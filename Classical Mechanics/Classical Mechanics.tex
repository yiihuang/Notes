\documentclass[10pt]{article}
\pdfoutput=1
%\usepackage{NotesTeX,lipsum}
\usepackage{NotesTeX,lipsum}
%\usepackage{showframe}

\title{\begin{center}{\Huge \textit{Notes}}\\{{\itshape Classical Electrodynamics}}\end{center}}
\author{Yi Huang}


\affiliation{
University of Minnesota
}

\emailAdd{yihphysics@gmail.com}

\begin{document}
	\maketitle
	\flushbottom
	\newpage
	\pagestyle{fancynotes}
	\part{Classical Mechanics}
	\section{Spring 2019}\label{sec:spring2019}
	\begin{margintable}\vspace{.8in}\footnotesize
		\begin{tabularx}{\marginparwidth}{|X}
		Section~\ref{sec:spring2019}. spring 2019\\
		\end{tabularx}
	\end{margintable}

\subsection{Problem 1 section 11 in Landau}
The energy of this system is
\begin{equation}
	E = \frac{1}{2} m l^2 \dot{\phi}^2 - mgl\cos{phi} = -mgl\cos{\phi_0},
\end{equation}
where $\phi_0$ is the maximum angle of motion. Separate variables
\begin{equation}
	\dd{t} = \sqrt{\tfrac{l}{2g}} \int \frac{\dd{\phi}}{\sqrt{\cos{\phi} - \cos{\phi_0}}},
\end{equation}
By symmetry, the period of motion is four times the time from angle $\phi = 0$ to $\phi = \phi_0$.
\begin{align*}
	T &= 4 \sqrt{\tfrac{l}{2g}} \int_0^{\phi_0} \frac{\dd{\phi}}{\sqrt{\cos{\phi} - \cos{\phi_0}}}\\
	  &= 2 \sqrt{\tfrac{l}{g}} \int_0^{\phi_0} \frac{\dd{\phi}}{\sqrt{\sin^2{\thalf \phi_0} - \sin^2{\thalf \phi}}}\\
		(\sin{\xi} = \frac{\sin{\thalf \phi}}{\sin{\thalf \phi_0}})
		&=  2 \sqrt{\tfrac{l}{g}} \int_0^{\phi_0} \frac{\dd{\phi}}{\sin{\thalf \phi_0} \sqrt{1 - \sin^2{\xi}}}\\
		&= 4 \sqrt{\tfrac{l}{g}} \int_0^{\pi/2} \frac{\dd{\xi}}{\sqrt{1 - \sin^2{\thalf \phi_0} \sin^2{\xi}}}\\
		&= 4 \sqrt{l/g} K\qty(\sin{\thalf \phi_0}).
\end{align*}
where we use $\cos{\phi} = 1-2\sin^2{\thalf \phi}$ in the third step, and change the variable in the 4th step s.t.
\begin{equation}
	\dd{\phi} = \frac{2 \sin{\thalf \phi_0} \cos{\xi}}{\cos{\thalf \phi}} \dd{\xi} = \frac{2 \sin{\thalf \phi_0} \cos{\xi}}{\sqrt{1 - \sin^2{\thalf \phi_0} \sin^2{\xi}}} \dd{\xi},
\end{equation}
in the last step we use the definition of complete elliptic integral of the first kind
\begin{equation}
	K(k) = \int_0^{\pi/2} \frac{\dd{x}}{\sqrt{1-k^2 \sin^2{x}}}.
\end{equation}

\subsection{Problem 2 section 11 in Landau}
The potential energy is
\begin{equation}
	U = -U_0/\cosh^2{\alpha x}\qc U_0 >0.
\end{equation}
The shape of this potential can be inferred by its limit
\begin{align*}
	\lim_{x \to 0} U(x) &= -U_0, \\
	\lim_{x \to \pm \infty} U(x) &= 0^-,
\end{align*}
which is like an attractive potential well centered at $x = 0$ with minimum $-U_0$, and approches zero when $x \to \pm \infty$.
The total energy $E$ satisfy
\begin{equation}
	-U_0 < E < 0,
\end{equation}
which means the particle is bounded by potential $U(x)$. The positive turning point is
\begin{equation}
	x_t = \cosh^{-1}{\sqrt{U_0/\abs{E}}}.
\end{equation}
Hence the period is
\begin{align*}
	T &= 4 \sqrt{m/2} \int_0^{x_t} \frac{\dd{x}}{\sqrt{E + U_0/\cosh^2{\alpha x}}} \\
	&= 2 \sqrt{2m} \int_0^{x_t} \frac{\cosh{\alpha x}\dd{x}}{\sqrt{U_0 - \abs{E} \cosh^2{\alpha x}}} \\
	&= \frac{2 \sqrt{2m}}{\alpha} \int \frac{\dd{\sinh{\alpha x}}}{\sqrt{U_0 - \abs{E}\qty(1+ \sinh^2{\alpha x})}} \\
	&= \frac{2}{\alpha} \sqrt{\frac{2m}{\abs{E}}} \int_0^1 \frac{\dd(\eta \sinh{\alpha x})}{\sqrt{1 - \eta^2\sinh^2{\alpha x}}} \\
	&= \frac{2}{\alpha} \sqrt{\frac{2m}{\abs{E}}} \int_0^1 \frac{\dd{u}}{\sqrt{1-u^2}} \\
	&= \frac{2 }{\alpha} \sqrt{\frac{2m}{\abs{E}}} \eval{\arcsin{u}}^1_0 \\
	&= \frac{\pi}{\alpha} \sqrt{\frac{2m}{\abs{E}}}
\end{align*}
where
\begin{equation}
	\eta = \sqrt{\tfrac{\abs{E}}{U_0 - \abs{E}}}
\end{equation}

\subsection{Kepler's problem, section 15 in Landau}
Given an attractive potential
\begin{equation}
	U(r) = -\frac{\alpha}{r},
\end{equation}
The motion of the particle is in a plane, which is defined by its initial velocity and the centrifugal force. We can write its Lagrangian
\begin{equation}
	L = \frac{1}{2} m (\dot{r}^2 + r^2 \dot{\phi}^2) + \frac{\alpha}{r}.
\end{equation}
There's a cyclic coordinate $\phi$, so the angular momentum is conserved
\begin{equation}
	\dv{t} \underbrace{\pdv{L}{\dot{\phi}}}_{M} = \dv{t}(m r^2 \dot{\phi}) = \pdv{L}{\phi} = 0,
\end{equation}
where $M$ is the angular momentum of the particle. This point can also be seen from the fact that $U(r)$ is spherically symmetric, so the angular momentum must be conserved.
The energy of this particle is
\begin{equation}
	E = \frac{1}{2} m \dot{r}^2 + \frac{M}{2m r^2} - \frac{\alpha}{r} = \frac{1}{2} m \dot{r}^2 + U_{\text{eff}}(r),
\end{equation}
where the effective potential is
\begin{equation}
	U_{\text{eff}}(r) = \frac{M}{2m r^2} - \frac{\alpha}{r}.
\end{equation}
From this we have the differential relation between $\dd{t}$ and $\dd{r}$
\begin{equation}
	\dd{t} = \sqrt{\frac{m}{2}} \frac{\dd{r}}{\sqrt{E - U_{\text{eff}}(r)}}.
\end{equation}
Also using the definition of angular momentum we have
\begin{equation}
	m r^2 \dv{\phi}{t} = M \Longleftrightarrow \dd{t} = \frac{m r^2}{M} \dd{\phi},
\end{equation}
Hence we obtain the differential equation for orbital
\begin{equation}
	\dd{\phi} = \sqrt{\frac{M^2}{2m}} \frac{\dd{r}/r^2}{\sqrt{E - U_{\text{eff}}(r)}}.
\end{equation}
Change variable $u = 1/r$ we have
\begin{equation}
	\dd{u} = -\dd{r}/r^2\qc U_{\text{eff}}(r) = \frac{M^2}{2m} u^2 - \alpha u,
\end{equation}
so we have the following integral
\begin{equation}
	\phi = - \sqrt{\tfrac{M^2}{2m}} \int \frac{\dd{u}}{\sqrt{E + \alpha u - \tfrac{M^2}{2m} u^2}}.
\end{equation}
Change the variable again to complete the square in the denorminator
\begin{equation}
	y = u - \frac{\alpha m }{M^2}\qc z = y \sqrt{\frac{\tfrac{M^2}{2m}}{E + \tfrac{\alpha^2 m }{2M^2}}},
\end{equation}
we have
\begin{align*}
	\phi &= - \sqrt{\tfrac{M^2}{2m}} \int \frac{\dd{y}}{\sqrt{E + \tfrac{\alpha^2 m }{2M^2} - \tfrac{M^2}{2m}y^2}} \\
	&= - \int \frac{\dd{z}}{\sqrt{1-z^2}}\\
	&= \arccos{z} + \const.
\end{align*}
This is the result in Landau.

\subsection{Two-body problem}
The Lagrangian of a two-body system in an inertial frame $K$ is as follows
\begin{equation}
	L = \frac{1}{2}m_1 v_1^2 + \frac{1}{2}m_2 v_2^2 - U(\vb{x}_1 - \vb{x}_2),
\end{equation}
where $\vb{x}_{1,2}$ are the coordinates of $m_{1,2}$ respectively, and the potential only depends on the relative position of the two bodies $\vb{x}_1 - \vb{x}_2$. This problem can be simplified by changing the variables from $\vb{x}_{1,2}$ to $\vb{R}$ and $\vb{r}$, where $\vb{R}$ is the center of mass of the two bodies, and $\vb{r}$ is the relative position
\begin{equation}
	\vb{R} = \frac{m_1 \vb{x}_1 + m_2 \vb{x}_2}{m_1 + m_2}\qc \vb{r} = \vb{x}_1 - \vb{x}_2.
\end{equation}
or inversely
\begin{align}
	\vb{x}_1 &= \vb{R} + \frac{m_2}{m_1 + m_2} \vb{r} = \vb{R} + \frac{\mu}{m_1} \vb{r}, \\
	\vb{x}_2 &= \vb{R} - \frac{m_1}{m_1 + m_2} \vb{r} = \vb{R} - \frac{\mu}{m_2} \vb{r},
\end{align}
where $\mu$ is the reduced mass
\begin{equation}
	\mu = \frac{m_1 m_2}{m_1 + m_2}.
\end{equation}
Hence after a little calculation, we can rewrite our Lagrangian as follows
\begin{equation}
	L = \frac{1}{2}(m_1 + m_2) V^2 + \frac{1}{2}\mu v^2 - U(\vb{r}),
\end{equation}
where $\vb{V} = \dot{\vb{R}}$, and $\vb{v} = \dot{\vb{r}}$. If there's no external force, the total momentum $\vb{P} = m_1 \vb{v}_1 + m_2 \vb{v}_2 = (m_1 + m_2) \vb{V}$ of this system is conserved, i.e. $\vb{V}$ is a constant vector.
One remark is that this Lagrangian is still in the reference frame $K$, and all we did is changing the variables from $\vb{x}_{1,2}$ to $\vb{R}$ and $\vb{r}$. From this Lagrangian, we can interpret the original two-body problem as a free particle with mass $(m_1 + m_2)$ and the second particle with mass $\mu$ inside a potential $U(\vb{r})$.

If we go to the center-of-mass frame $K'$ of the two bodies, then all we need to modify for the above Lagrangian is to set $\vb{R}$ as the origin, and the speed of reference frame $K'$ relative to $K$ is $\vb{V}$, which is a constant vector as mentioned before. Therefore $K'$ is also an inertial frame. In this frame the coordinates of the two particles are colinear in $\vb{r}$
\begin{align}
	\vb{x'}_1 &= \frac{m_2}{m_1 + m_2} \vb{r} = \frac{\mu}{m_1} \vb{r}, \label{eq: 1.4.1}\\
	\vb{x'}_2 &= \frac{m_1}{m_1 + m_2} \vb{r} = - \frac{\mu}{m_2} \vb{r}, \label{eq: 1.4.2}
\end{align}
For the velocity in $K'$ we have
\begin{align}
	\vb{V} &= \vb{V'} + \vb{V} \Longrightarrow \vb{V'} = 0, \\
	\vb{v} &= \vb{v'},
\end{align}
because $\vb{r'} = \vb{r}$ is invariant under this reference frame transformation. The Lagragian in $K'$
\begin{align*}
	L' &= \frac{1}{2}(m_1 + m_2) V'^2 + \frac{1}{2}\mu v'^2 - U(\vb{r'}) \\
	&= \frac{1}{2}\mu v^2 - U(\vb{r}).
\end{align*}
Therefore we conclude that the center of the potential $U(\vb{r})$ coincides with center of mass of the two-body system. The motions of the two bodies in frame $K'$ can be think of an effective one-body motion $\vb{r}(t)$, and their coordinates are given by \eqref{eq: 1.4.1} and  \eqref{eq: 1.4.2}.



\end{document}
