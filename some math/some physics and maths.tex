\documentclass[10pt]{article}
\pdfoutput=1
%\usepackage{NotesTeX,lipsum}
\usepackage{NotesTeX,lipsum}
%\usepackage{showframe}

\title{\begin{center}{\Huge \textit{Notes}}\\{{\itshape Quantum Mechanics}}\end{center}}
\author{Yi Huang\footnote{\href{https://yiihuang.com/}{\textit{My Personal Website}}}}


\affiliation{
University of Minnesota
}

\emailAdd{yihphysics@gmail.com}

\begin{document}
	\maketitle
	\flushbottom
	\newpage
	\pagestyle{fancynotes}
	\part{Caprice}
	\section{Fall 2018}\label{sec:fall2018}
	\begin{margintable}\vspace{.8in}\footnotesize
		\begin{tabularx}{\marginparwidth}{|X}
		Section~\ref{sec:fall2018}. Fall 2018\\
		\end{tabularx}
	\end{margintable}

	This is my note for some non-trivial but not systematic problems which involves some interesting physics or maths.

	\subsection{Walkway equilibrium}

	Suppose the mass of the objects attached to each end of the rope are $m_1$ and $m_2$, The angles between each segment of the rope, bended by the central object which has mass $M$, with the horizontal plane are $\theta$ and $\phi$. The distance between two pulleys is $L$, and what we want to know is the vertical displacement $d$ of the central object. Thus we can obtaind the equations for $d$ when the system is at equilibrium.

	\begin{gather}
		L = d (\cot \theta + \cot \phi), \\
		m_1 g \cos \theta = m_2 \cos \phi, \\
		m_1 g \sin \theta + m_2 g \sin \phi = Mg, 
	\end{gather}




\end{document}
