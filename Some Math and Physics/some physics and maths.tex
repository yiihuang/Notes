\documentclass[10pt]{article}
\pdfoutput=1
%\usepackage{NotesTeX,lipsum}
\usepackage{NotesTeX,lipsum}
%\usepackage{showframe}

\title{\begin{center}{\Huge \textit{Notes}}\\{{\itshape Something}}\end{center}}
\author{Yi Huang\footnote{\href{https://yiihuang.com/}{\textit{My Personal Website}}}}


\affiliation{
University of Minnesota
}

\emailAdd{yihphysics@gmail.com}

\begin{document}
	\maketitle
	\flushbottom
	\newpage
	\pagestyle{fancynotes}
	\part{Caprice}
	\section{Fall 2018}\label{sec:fall2018}
	\begin{margintable}\vspace{.8in}\footnotesize
		\begin{tabularx}{\marginparwidth}{|X}
		Section~\ref{sec:fall2018}. Fall 2018\\
		\end{tabularx}
	\end{margintable}

	This is my note for some non-trivial but not systematic problems which involves some interesting physics or maths.

	\subsection{Walkway equilibrium}

	Suppose the mass of the objects attached to each end of the rope are $m_1$ and $m_2$, The angles between each segment of the rope, bended by the central object which has mass $M$, with the horizontal plane are $\theta$ and $\phi$. The distance between two pulleys is $L$, and what we want to know is the vertical displacement $d$ of the central object. Thus we can obtaind the equations for $d$ when the system is at equilibrium.
	\begin{gather}
		L = d (\cot \theta + \cot \phi), \label{eq: total L}\\
		m_1 g \cos \theta = m_2 \cos \phi, \label{eq: Tx}\\
		m_1 g \sin \theta + m_2 g \sin \phi = Mg, \label{eq: Ty}
	\end{gather}
	From \eqref{eq: Tx}, we have $\cos \phi = \tfrac{m_1}{m_2} \cos \theta$, thus \eqref{eq: Ty} can be written as
	\begin{equation}
		m_1 \sin \theta + m_2 \sqrt{1 - \tfrac{m_1^2}{m_2^2}(1- \sin^2 \theta)} = M,
	\end{equation}
	such that we can solve for $\sin \theta$ and $\cos \theta$
	\begin{gather}
		\sin \theta = \frac{M^2 + m_1^2 - m_2^2}{2 M m_1}, \\
		\cos \theta = \sqrt{1 - \sin^2 \theta} = \frac{1}{2 M m_1} \sqrt{[(m_1 + m_2)^2 - M^2][M^2 - (m_1 - m_2)^2]}, \\
		\cot \theta = \frac{\sqrt{[(m_1 + m_2)^2 - M^2][M^2 - (m_1 - m_2)^2]}}{M^2 + m_1^2 - m_2^2},
	\end{gather}
	together with $\sin \phi$ and $\cos \phi$
	\begin{gather}
		\cos \phi = \frac{m_1}{m_2} \cos \theta = \frac{1}{2 M m_2} \sqrt{[(m_1 + m_2)^2 - M^2][M^2 - (m_1 - m_2)^2]}, \\
		\sin \phi = \sqrt{1 - \cos^2 \phi} = \frac{M^2 - m_1^2 + m_2^2}{2M m_2}, \\
		\cot \phi = \frac{\sqrt{[(m_1 + m_2)^2 - M^2][M^2 - (m_1 - m_2)^2]}}{M^2 - m_1^2 + m_2^2}.
	\end{gather}
	Therefore we can plug into \eqref{eq: total L} and obtain the expression of $d$ as follows
	\begin{equation}
		d = \frac{L[M^4 - (m_1^2 - m_2^2)^2]}{2M^2 \sqrt{[(m_1 + m_2)^2 - M^2][M^2 - (m_1 - m_2)^2]}}.
	\end{equation}
	The equilibrium condition in this case is
	\begin{equation}
		\abs{m_1 - m_2} < M < (m_1 + m_2).
	\end{equation}
	such that the argument under the square root is positive. Also we can easily check that if $m_1 = m_2 = m$ then this result reduces to our former result
	\begin{equation}
		d = \frac{L M}{2\sqrt{4m^2 - M^2}}.
	\end{equation}

	\subsection{A derivation of Gamma function from Fourier transfrom}

	It is well-known that $\Gamma(n+1) = n!$ for any natural number $n \in \mathbb{N}$. It is natural to ask what is $\Gamma(x)$ for any real number $x \ge 1$. Our purpose is to show that Gamma function can be express as an integral
	\begin{equation}
		\Gamma(x) = \int_0^{\infty} \dd{x} t^{x-1} \e^t, \label{eq: gamma integral}
	\end{equation}
	given that
	\begin{equation}
		\Gamma(x+1) = x \Gamma(x), \label{eq: gamma recursion}
	\end{equation}
	which is the most essential property and motivation to define the Gamma function.

	Using Talor expansion we can easily show that
	\begin{equation}
		f(x + \Delta x) = \sum_{n=0}^{\infty} \frac{f^{(n)}(x)}{n!} (\Delta x)^n = \exp(\Delta x \dv{x}) f(x)
	\end{equation}
	where $\Delta x$ is a constant of translation. Therefore $\Gamma(x + 1) = \e^{\dv{x}} \Gamma(x)$, and we can rewrite \eqref{eq: real gamma recursion} as
	\begin{equation}
		\e^{\dv{x}} \Gamma(x) = x \Gamma(x). \label{eq: real gamma recursion}
	\end{equation}
	Consider doing Fourier transform of \eqref{eq: real gamma recursion}, such that $\dv{x} \to i \omega$, $x \to i \dv{\omega}$, $\Gamma(x) \to \tilde{\Gamma}(\omega)$, and
	\begin{gather}
		\tilde{\Gamma}(\omega) = \mathcal{F}[\Gamma(x)] = \int_{-\infty}^{\infty} \dd{x} \Gamma(x) \e^{-i \omega x}, \\
		\Gamma(x) = \mathcal{F}^{-1}[\tilde{\Gamma}(\omega)] = \int_{-\infty}^{\infty} \frac{\dd{\omega}}{2 \pi} \tilde{\Gamma}(\omega) \e^{i \omega x}, \\
		\e^{i \omega} \tilde{\Gamma}(\omega) = i \dv{\omega} \tilde{\Gamma}(\omega).
	\end{gather}
	Solve the above differential equation of $\tilde{\Gamma}(\omega)$ we find
	\begin{gather}
		\tilde{\Gamma}(\omega) = C \exp(- \e^{i \omega}), \label{eq: real gamma} \\
		\Gamma(x) = \int_{-\infty}^{\infty}  \frac{\dd{\omega}}{2 \pi} C \exp(- \e^{i \omega}) \e^{i \omega x}. \label{eq: inverse fourier gamma in real}
	\end{gather}
	However, \eqref{eq: inverse fourier gamma in real} does not converge, since \eqref{eq: real gamma} is a nonzero periodic function.

	To resolve this difficulty of convergence, we expand the domain of Gamma function to the complex plane, such that $\Gamma(z + 1) = z \Gamma(z)$, where $z \in \mathbb{C}$. Consider a pure imaginary number $z = i x$, where $x \in \mathbb{R}$, we can rewrite the recursion relation \eqref{eq: gamma recursion} as
	\begin{equation}
		\e^{-i \dv{x}} \Gamma(ix) = ix \Gamma(ix)
	\end{equation}
	Again using Fourier transform we have
	\begin{equation}
		\e^{\omega} \mathcal{F}[\Gamma(ix)] = - \dv{\omega} \mathcal{F}[\Gamma(ix)], \label{eq: pure imaginary gamma}
	\end{equation}
	where $\mathcal{F}[\Gamma(ix)]$ is the Fourier transform of $\Gamma(ix)$
	\begin{equation}
		\mathcal{F}[\Gamma(ix)] = \int_{-\infty}^{\infty} \dd{x} \Gamma(ix) \e^{- i \omega x}.
	\end{equation}
	Solve \eqref{eq: pure imaginary gamma} we have
	\begin{gather}
		\mathcal{F}[\Gamma(ix)] = C \exp(- \e^{\omega}), \\
		\Gamma(ix) = \frac{C}{2\pi} \int_{-\infty}^{\infty} \dd{\omega} \exp(- \e^{\omega}) \e^{i \omega x}.
	\end{gather}
	Thus
	\begin{align*}
		\Gamma(z) &= \frac{C}{2\pi} \int_{-\infty}^{\infty} \dd{\omega} \exp(- \e^{\omega}) \e^{\omega z} \\
		&= \frac{C}{2\pi} \int_{-\infty}^{\infty} \dd{\e^{\omega}} \exp(- \e^{\omega}) \e^{\omega (z-1)} \\
		&= \frac{C}{2\pi} \int_{-\infty}^{\infty} \dd{t} t^{z-1} \e^{-t}.
	\end{align*}
	To determine the constant $C$ we use the fact that $\Gamma(1) = 0! = 1$, thus $C/2\pi = 1$, and we obtain the final integral expression of Gamma function
	\begin{equation}
		\Gamma(z) = \int_{-\infty}^{\infty} \dd{t} t^{z-1} \e^{-t}
	\end{equation}
	where $z \in \mathbb{C}$.

	\subsection{Euler's reflection formula}

	In mathematics, a reflection formula or reflection relation for a function f is a relationship between $f(a-x)$ and $f(x)$. A famous relationship is Euler's reflection formula
	\begin{proposition}\label{prop: Euler's reflection formula}
		\begin{equation}
			\Gamma(z)\Gamma(1-z) = \frac{\pi}{\sin{\pi z}}\qc z \not \in \mathbb{Z}.
		\end{equation}
	\end{proposition}
	\begin{proof}
		To prove this reflection formula, we first notice a relationship between Gamma function and Beta function
		\begin{equation}
			B(q, p) = \frac{\Gamma(q) \Gamma(p)}{\Gamma(q + p)},
		\end{equation}
		where
		\begin{equation}
			B(q, p) = \int_0^1 \dd{t} t^{q-1} (1-t)^{p-1} \qc q,p \neq 0, -1, -2, \dots
		\end{equation}
		This can be shown by performing a variable transformation of $\Gamma(q) \Gamma(p)$
		\begin{align*}
			\Gamma(q) \Gamma(p) &= \int_0^{\infty} \dd{u} \e^{-u} u^{q-1} \int_0^{\infty} \dd{v} \e^{-v} v^{p-1} \\
			u = zt, v = z(1-t)\qc &= \int \dd{z} \dd{t} \abs{\pdv{(u,v)}{(z,t)}} \e^{-z} (zt)^{q-1} [z(1-t)]^{p-1} \\
			&= \int_0^{\infty} \dd{z} z^{q+p-1} \e^{-z} \int_0^1 \dd{t} t^{q-1} (1-t)^{p-1} \\
			&= \Gamma(q+p) B(q, p).
		\end{align*}
		Therefore
		\begin{equation}
			\Gamma(z)\Gamma(1-z) = B(z, 1-z) = \int_0^1 \dd{t} t^{z-1} (1-t)^{-z}.
		\end{equation}
		In order to prove Proposition \ref{prop: Euler's reflection formula}, we only need to prove
		\begin{equation}
			\int_0^1 \dd{t} t^{z-1} (1-t)^{-z} = \frac{\pi}{\sin{\pi z}}.
		\end{equation}
		Perform a variable substitution $t \to \frac{x}{1+x}$, such that $\dd{t} = \dd{x}/(1+x)^2$ and
		\begin{equation}
			\int_0^1 \dd{t} t^{z-1} (1-t)^{-z} &= \int_0^{\infty} \dd{x} \frac{x^{z-1}}{1+x}
		\end{equation}
		Consider the following integral
		\begin{equation}
			\int_0^{\infty} \dd{x} \frac{x^{\alpha-1}}{x+\e^{i \phi}}\qc 0<\alpha<1\qc -\pi<\phi<\pi.
		\end{equation}
	\end{proof}

\end{document}
