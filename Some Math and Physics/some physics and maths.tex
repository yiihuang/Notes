\documentclass[10pt]{article}
\pdfoutput=1
%\usepackage{NotesTeX,lipsum}
\usepackage{NotesTeX,lipsum}
%\usepackage{showframe}

\title{\begin{center}{\Huge \textit{Notes}}\\{{\itshape Something}}\end{center}}
\author{Yi Huang\footnote{\href{https://yiihuang.com/}{\textit{My Personal Website}}}}


\affiliation{
University of Minnesota
}

\emailAdd{yihphysics@gmail.com}

\begin{document}
	\maketitle
	\flushbottom
	\newpage
	\pagestyle{fancynotes}
	\part{Caprice}
	\section{Fall 2018}\label{sec:fall2018}
	\begin{margintable}\vspace{.8in}\footnotesize
		\begin{tabularx}{\marginparwidth}{|X}
		Section~\ref{sec:fall2018}. Fall 2018\\
		\end{tabularx}
	\end{margintable}

	This is my note for some non-trivial but not systematic problems which involves some interesting physics or maths.

	\subsection{Walkway equilibrium}

	Suppose the mass of the objects attached to each end of the rope are $m_1$ and $m_2$, The angles between each segment of the rope, bended by the central object which has mass $M$, with the horizontal plane are $\theta$ and $\phi$. The distance between two pulleys is $L$, and what we want to know is the vertical displacement $d$ of the central object. Thus we can obtaind the equations for $d$ when the system is at equilibrium.
	\begin{gather}
		L = d (\cot \theta + \cot \phi), \label{eq: total L}\\
		m_1 g \cos \theta = m_2 \cos \phi, \label{eq: Tx}\\
		m_1 g \sin \theta + m_2 g \sin \phi = Mg, \label{eq: Ty}
	\end{gather}
	From \eqref{eq: Tx}, we have $\cos \phi = \tfrac{m_1}{m_2} \cos \theta$, thus \eqref{eq: Ty} can be written as
	\begin{equation}
		m_1 \sin \theta + m_2 \sqrt{1 - \tfrac{m_1^2}{m_2^2}(1- \sin^2 \theta)} = M,
	\end{equation}
	such that we can solve for $\sin \theta$ and $\cos \theta$
	\begin{gather}
		\sin \theta = \frac{M^2 + m_1^2 - m_2^2}{2 M m_1}, \\
		\cos \theta = \sqrt{1 - \sin^2 \theta} = \frac{1}{2 M m_1} \sqrt{[(m_1 + m_2)^2 - M^2][M^2 - (m_1 - m_2)^2]}, \\
		\cot \theta = \frac{\sqrt{[(m_1 + m_2)^2 - M^2][M^2 - (m_1 - m_2)^2]}}{M^2 + m_1^2 - m_2^2},
	\end{gather}
	together with $\sin \phi$ and $\cos \phi$
	\begin{gather}
		\cos \phi = \frac{m_1}{m_2} \cos \theta = \frac{1}{2 M m_2} \sqrt{[(m_1 + m_2)^2 - M^2][M^2 - (m_1 - m_2)^2]}, \\
		\sin \phi = \sqrt{1 - \cos^2 \phi} = \frac{M^2 - m_1^2 + m_2^2}{2M m_2}, \\
		\cot \phi = \frac{\sqrt{[(m_1 + m_2)^2 - M^2][M^2 - (m_1 - m_2)^2]}}{M^2 - m_1^2 + m_2^2}.
	\end{gather}
	Therefore we can plug into \eqref{eq: total L} and obtain the expression of $d$ as follows
	\begin{equation}
		d = \frac{L[M^4 - (m_1^2 - m_2^2)^2]}{2M^2 \sqrt{[(m_1 + m_2)^2 - M^2][M^2 - (m_1 - m_2)^2]}}.
	\end{equation}
	The equilibrium condition in this case is
	\begin{equation}
		\abs{m_1 - m_2} < M < (m_1 + m_2).
	\end{equation}
	such that the argument under the square root is positive. Also we can easily check that if $m_1 = m_2 = m$ then this result reduces to our former result
	\begin{equation}
		d = \frac{L M}{2\sqrt{4m^2 - M^2}}.
	\end{equation}

\end{document}
