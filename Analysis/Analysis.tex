\documentclass[10pt]{article}
\pdfoutput=1
%\usepackage{NotesTeX,lipsum}
\usepackage{NotesTeX,lipsum}
%\usepackage{showframe}

\title{\begin{center}{\Huge \textit{Notes}}\\{{\itshape Analysis}}\end{center}}
\author{Yi Huang\footnote{\href{https://yiihuang.com/}{\textit{My Personal Website}}}}


\affiliation{
University of Minnesota
}

\emailAdd{yihphysics@gmail.com}

\begin{document}
	\maketitle
	\flushbottom
	\newpage
	\pagestyle{fancynotes}
	\part{Caprice}
	\section{Fall 2018}\label{sec:fall2018}
	\begin{margintable}\vspace{.8in}\footnotesize
		\begin{tabularx}{\marginparwidth}{|X}
		Section~\ref{sec:fall2018}. Fall 2018\\
		\end{tabularx}
	\end{margintable}

	This is my note for analysis. The textbook I use are Stein's classic four volumes on analysis.

	\subsection{Cauchy criterion}

	\begin{definition}
		If $(a_n)$ is a sequence, then $\lim_{n \to \infty} a_n = L$ means that for every $\veps > 0$ there exists a corresponding $N \in \mathbb{N}^+$ such that if $n \ge N$, then $\abs{a_n - L} < \veps$. If this limit exists, then we say that the sequence $(a_n)$ converges, and if this limit doesn't exists then we say the sequence $(a_n)$ diverges.
	\end{definition}
	\begin{proposition}
		A converging sequence of $\mathbb{R}$ has a unique limit.
	\end{proposition}
	\begin{proof}
		Suppose both $L_1$ and $L_2$ are limits of $(a_n)$ and $L_1 \neq L_2$, which means for any $\veps > 0$, there exists $N_1 \in \mathbb{N}^+$ such that
		\begin{equation}
			\abs{a_n - L_1}, \qq{for every} n \ge N_1;
		\end{equation}
		and for any $\veps > 0$, there exists $N_2 \in \mathbb{N}^+$ such that
		\begin{equation}
			\abs{a_n - L_2}, \qq{for every} n \ge N_2.
		\end{equation}
		Since $L_1 \neq L_2$, we can choose $\epsilon = \abs{L_1 - L_2}/4$, $N = \max \{ N_1, N_2 \}$, then
		\begin{align*}
			\abs{L_1 - L_2} &= \abs{(a_n - L_2) + (L_1 - a_n)} \\
			&\le \abs{a_n - L_1} + \abs{a_n - L_2} \\
			&\le 2 \veps = \abs{L_1 - L_2}/2, \qq{for every} n \ge N,
		\end{align*}
		which can be true only if $L_1 = L_2$, and this contradicts to the assumption $L_1 \neq L_2$. Therefore we prove $L_1 = L_2$ by contradiction.
	\end{proof}

	\begin{definition}
		A sequence $(a_n)$ is said to be a Cauchy sequence iff for any $\veps >0$, there exists $N \ge \mathbb{N}^+$, such that
		\begin{equation}
			\abs{a_n - a_m} < \veps,
		\end{equation}
		for all $n,m \ge N$. In other words, a Cauchy sequence is one inwhich the terms eventually cluster together.
	\end{definition}
	\begin{theorem}\label{thm: cauchy criterion}
		(Cauchy Criterion.) A sequence is Cauchy iff it converges.
	\end{theorem}
	\begin{proof}
		Sufficiency: We need to prove if a sequence converges, it is Cauchy. Given any $\veps > 0$. Let
		\begin{equation}
			L := \lim_{n \to \infty} a_n.
		\end{equation}
		There exists $N$ such that for all $n \ge N$,
		\begin{equation}
			\abs{a_n - L} < \frac{\veps}{2}.
		\end{equation}
		Thus, for all $m,n \ge N$,
		\begin{equation}
			\abs{a_n - a_m} = \abs{(a_n - L) + (L - a_m)} \le \abs{a_n - L} + \abs{a_m - L} \le \veps.
		\end{equation}

		Necessity: We need to show a Cauchy sequence must converge. The proof proceeds in several steps, which we isolate and prove subsequently.
		\begin{steps}
			\item Since $(a_n)$ is Cauchy, it must be bounded.
			\item Since $(a_n)$ is bounded, it has a convergent subsequence $(a_{n_k})$. Let $x:= \lim_{k \to \infty} a_{n_k}$.
			\item Show that $\lim_{n \to \infty} = x$.
		\end{steps}
	\end{proof}

	\begin{proposition}
		(Step 1 in the proof of theorem \ref{thm: cauchy criterion}.) If a sequence $(a_n)$ is Cauchy, then it is bounded.
	\end{proposition}
	\begin{proof}
		There exits $N$ such that $\abs{a_n - a_m} < 1$ for all $n,m \ge N$. In particular, we have $\abs{a_n - a_m} < 1$ for all $n \ge N$, whence
		\begin{equation}
			a_n \in (a_N - 1, a_N + 1)
		\end{equation}
		for all $n \ge N$. Thus $\{ a_n: n \ge N \}$ is bounded. Also, $\{ a_n: n < N \}$ is bounded since it is finite. We conclude that the entire range of the sequence $(a_n)$ is bounded.
	\end{proof}

	Before we prove step 2 of theorem \ref{thm: cauchy criterion}, we first prove three theorems as lemmas: monotone subsequence theorem, boundedness of convergent sequences, and monotone convergence theorem.
	\begin{lemma}\label{lm: monotone subsequence theorem}
		(Monotone Subsequence Theorem.) Every sequence has a momotone subsequence.
	\end{lemma}
	\begin{proof}
		Let $(a_n)$ denote a sequence. We call a term $a_k$ a peak iff $a_k \ge a_m$ for all $m > k$. There are two cases:
		\begin{enumerate}
			\item There are infinitely many peaks.
			\item There are finitely many peaks.
		\end{enumerate}
		In the first case, the subsequence consisting of the peaks forms a monotonically decreasing sequence. In the second case, there exists some largest $M$ such that $a_M$ is a peak. Let $m_1 = M+1$. Since $a_{m_1}$ isn't a peak, there exists $m_2 > m_1$ such that $a_{m_1} < a_{m_2}$. Since $a_{m_2}$ isn't a peak, there exists $m_3 > m_2$ such that $a_{m_2} < a_{m_3}$. We can continue this process indefinitely, thus creating a monotonically increasing subsequence $(a_{m_k})$.
	\end{proof}
	\begin{lemma}\label{lm: boundedness of convergent sequence}
		(Boundedness of Convergent Sequences.) If $(a_n)$ is a sequence of real numbers that is convergent to $L \in \mathbb{R}$
		\begin{equation}
			\lim_{n \to \infty} a_n = L,
		\end{equation}
		then $(a_n)$ is also bounded.
	\end{lemma}
	\begin{proof}
		Let $\veps_0 = 1$. There exists $N \in \mathbb{N}^+$ such that if $n \ge N$ then $\abs{a_n - L} < \veps = 1$. Equivalently, $L-1 < a_n < L+1$ for all $n \ge N$, $\{ a_n: n \ge N \}$ is bounded. Also $\{ a_n: n \le N \}$ is bounded since it is finite.

		It is important to recongnize in the theorem above that we could no t have just take the maximum value from the start because we cannot take the maximum of an infinite set, only a finite set.

		It is also important to note that the converse of this theorem is not always true, that is just because a sequence $(a_n)$ is bounded does not imply $(a_n)$ to be convergent. For example, the sequence $((-1)^n) = (-1, 1, -1, 1, \dots)$ is bounded, however, clearly $((-1)^n)$ does not converge.
	\end{proof}
	\begin{lemma} \label{lm: monotone convergence theorem}
		(Monotone Convergence Theorem.) If $(a_n)$ is a monotone sequence of real numbers, then $(a_n)$ is convergent iff $(a_n)$ is bounded.
	\end{lemma}
	\begin{proof}
		Let $(a_n)$ be a monotone sequence.

		Necessity: If $(a_n)$ is convergent, then by the lemma \ref{lm: boundedness of convergent sequence} we just prove above, $(a_n)$ is bounded.

		Sufficiency: If $(a_n)$ is bounded, there are two cases to consider.

		Case 1. Suppose that $(a_n)$ is an increasing sequence that is there exists $M \in \mathbb{R}$ such that for any $n \in \mathbb{N}^+$ we have $a_n \le M$. By the completeness property of the real numbers, $\{a_n\}$ has a supremum in $\mathbb{R}$, call it $L = \sup\{ a_n \}$. Let $\veps > 0$ be given. Since $L$ is the supremum of $\{ a_n \}$, then $L - \veps$ is not an upper bound to $\{ a_n \}$ and so there exists $N$ such that $L - \veps < a_N$. Since $(a_n)$ is an increasing sequence, then for all $n \ge N$, we have $a_N \le a_n$, and so
		\begin{equation}
			L - \veps < a_N \le a_n \le L < L + \veps.
		\end{equation}
		Omitting the unnecessary parts of the inequality, we see that for $n \ge N$ we have $L-\veps < a_n < L+ \veps$, and so $\abs{a_n - L} < \veps$. By definition of limit, since $\veps > 0$ is arbitrary, we have that $\lim_{n \to \infty} a_n = L$, that is $(a_n)$ is convergent to $L$.

		Case 2. Suppose that $(a_n)$ is a decreasing sequence that is there exists $n \in \mathbb{R}$ such that for any $n \in \mathbb{N}^+$ we have $a_n \ge M$. By the completeness property of the real numbers, $\{a_n\}$ has a infimum in $\mathbb{R}$, call it $L = \inf\{ a_n \}$. Let $\veps > 0$ be given. Since $L$ is the infimum of $\{ a_n \}$, then $L + \veps$ is not a lower bound to $\{ a_n \}$ and so there exists $N$ such that $a_N < L + \veps$. Since $(a_n)$ is an decreasing sequence, then for all $n \ge N$, we have $a_n \le a_N$, and so
		\begin{equation}
			L - \veps < a_n \le a_N \le L < L + \veps.
		\end{equation}
		Omitting the unnecessary parts of the inequality, we see that for $n \ge N$ we have $L-\veps < a_n < L+ \veps$, and so $\abs{a_n - L} < \veps$. By definition of limit, since $\veps > 0$ is arbitrary, we have that $\lim_{n \to \infty} a_n = L$, that is $(a_n)$ is convergent to $L$.

		In both cases, the bounded sequence $(a_n)$ is convergent.
	\end{proof}
	\begin{corollary} \label{cor: mono bound}
		If $(a_n)$ is a monotone sequence that's bounded, then
		\begin{align}
			\lim_{n \to \infty} a_n &= \sup\{a_n\} \qq{if $(a_n)$ is increasing;} \\
			\lim_{n \to \infty} a_n &= \inf\{a_n\} \qq{if $(a_n)$ is decreasing.}
		\end{align}
	\end{corollary}
	The proof of corollary \ref{cor: mono bound} is done in lemma \ref{lm: monotone convergence theorem}.
	\begin{proposition}
		(Step 2 in the proof of theorem \ref{thm: cauchy criterion}.) Every bounded sequence has a convergent subsequence.
	\end{proposition}
	\begin{proof}
		Suppose $(a_n)$ is a bounded sequence. By the monotone subsequence theorem (lemma \ref{lm: monotone subsequence theorem}), it has a monotone subsequence $(a_{n_k})$. But then this subsequence is both bounded and monotone, whence it is convergent by the monotone convergence theorem (lemma \ref{lm: monotone convergence theorem}).
	\end{proof}
	Finally, we prove step 3 in the proof of theorem \ref{thm: cauchy criterion}.
	\begin{proposition}
		(Step 3 in the proof of theorem \ref{thm: cauchy criterion}.) If the subsequence of a Cauchy sequence converges to $x$ then the sequence itself converges to $x$.
	\end{proposition}
	\begin{proof}
		Let $(a_n)$ be a Cauchy sequence, and let $(a_{n_k})$ be a convergent subsequence. Set $x = \lim_{k \to infty} a_{n_k}$. We wish to prove that $a_n \to x$ as $n \to infty$.

		Given $\veps > 0$, there exists $K$ such that
		\begin{equation}
			\abs{a_{n_k} - x} < \veps / 2 \qq{for all $k \ge K$}.
		\end{equation}
		Also, since $(a_n)$ is Cauchy, there exists $N$ such that
		\begin{equation}
			\abs{a_n - a_m} < \veps /2 \qq{for all $n,m \ge N$.}
		\end{equation}
		Pick $l > K$ large enough so that $n_l > N$. Then for all $n > N$, we have
		\begin{align*}
			\abs{a_n - x} &= \abs{(a_n - a_{n_l}) + (a_{n_l} - x)} \\
			&\le \abs{(a_n - a_{n_l})} + \abs{(a_{n_l} - x)} \\
			&< \veps,
		\end{align*}
		which concludes the proof of theorem \ref{thm: cauchy criterion}.

	\end{proof}

\end{document}
