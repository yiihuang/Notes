\documentclass[10pt]{article}
\pdfoutput=1
%\usepackage{NotesTeX,lipsum}
\usepackage{NotesTeX,lipsum}
%\usepackage{showframe}

\title{\begin{center}{\Huge \textit{Notes}}\\{{\itshape Analysis}}\end{center}}
\author{Yi Huang\footnote{\href{https://yiihuang.com/}{\textit{My Personal Website}}}}


\affiliation{
University of Minnesota
}

\emailAdd{yihphysics@gmail.com}

\begin{document}
	\maketitle
	\flushbottom
	\newpage
	\pagestyle{fancynotes}
	\part{Caprice}
	\section{Fall 2018}\label{sec:fall2018}
	\begin{margintable}\vspace{.8in}\footnotesize
		\begin{tabularx}{\marginparwidth}{|X}
		Section~\ref{sec:fall2018}. Fall 2018\\
		\end{tabularx}
	\end{margintable}

	This is my note for analysis. The textbook I use are Stein's classic four volumes on analysis.

	\subsection{Cauchy criterion}

	\begin{definition}
		If $(a_n)$ is a sequence, then $\lim_{n \to \infty} a_n = L$ means that for every $\veps > 0$ there exists a corresponding $N \in \mathbb{N}^+$ such that if $n \ge N$, then $\abs{a_n - L} < \veps$. If this limit exists, then we say that the sequence $(a_n)$ converges, and if this limit doesn't exists then we say the sequence $(a_n)$ diverges.
	\end{definition}
	\begin{proposition}
		A converging sequence of $\mathbb{R}$ has a unique limit.
	\end{proposition}
	\begin{proof}
		Suppose both $L_1$ and $L_2$ are limits of $(a_n)$ and $L_1 \neq L_2$, which means for any $\veps > 0$, there exists $N_1 \in \mathbb{N}^+$ such that
		\begin{equation}
			\abs{a_n - L_1}, \qq{for every} n \ge N_1;
		\end{equation}
		and for any $\veps > 0$, there exists $N_2 \in \mathbb{N}^+$ such that
		\begin{equation}
			\abs{a_n - L_2}, \qq{for every} n \ge N_2.
		\end{equation}
		Since $L_1 \neq L_2$, we can choose $\epsilon = \abs{L_1 - L_2}/4$, $N = \max \{ N_1, N_2 \}$, then
		\begin{align*}
			\abs{L_1 - L_2} &= \abs{(a_n - L_2) + (L_1 - a_n)} \\
			&\le \abs{a_n - L_1} + \abs{a_n - L_2} \\
			&\le 2 \veps = \abs{L_1 - L_2}/2, \qq{for every} n \ge N,
		\end{align*}
		which can be true only if $L_1 = L_2$, and this contradicts to the assumption $L_1 \neq L_2$. Therefore we prove $L_1 = L_2$ by contradiction.
	\end{proof}
\end{document}
