\documentclass[10pt]{article}
\pdfoutput=1
%\usepackage{NotesTeX,lipsum}
\usepackage{NotesTeX,lipsum}
%\usepackage{showframe}

\title{\begin{center}{\Huge \textit{Notes}}\\{{\itshape Classical Electrodynamics}}\end{center}}
\author{Yi Huang}


\affiliation{
University of Minnesota
}

\emailAdd{yihphysics@gmail.com}

\begin{document}
	\maketitle
	\flushbottom
	\newpage
	\pagestyle{fancynotes}
	\part{Caprice}
	\section{Spring 2019}\label{sec:spring2019}
	\begin{margintable}\vspace{.8in}\footnotesize
		\begin{tabularx}{\marginparwidth}{|X}
		Section~\ref{sec:spring2019}. spring 2019\\
		\end{tabularx}
	\end{margintable}

\subsection{The angular integral in (4.16')}

Show the following equation is true
\begin{equation}
	\int \dd{\Omega} \vb{n} \cos{\gamma} = \frac{4 \pi \vb{n}'}{3},
\end{equation}
where $\vb{n} = \vb{i} \sin{\theta} \cos{\phi} + \vb{j} \sin{\theta} \sin{\phi} + \vb{k} \cos{\theta}$, and $\cos{\gamma} = \cos{\theta} \cos{\theta'} + \sin{\theta} \sin{\theta'} \cos(\phi - \phi')$.
\begin{proof}
	First look at the $x$ component of the integral:
	\begin{align*}
		\int \dd{\Omega} \sin{\theta} \cos{\phi} \cos{\gamma} &= \int \dd{\cos{\theta}} \dd{\phi} \sin{\theta} \cos{\phi} [cos{\theta} \cos{\theta'} + \sin{\theta} \sin{\theta'} \cos(\phi - \phi')] \\
		(\text{integrate over $\phi$}) &= \pi \int \dd{\cos{\theta}} \sin^2{\theta} \sin{\theta'} \cos{\phi'} \\
		(u = \cos{\theta}) &= \pi \sin{\theta'} \cos{\phi'} \int_{-1}^{1} \dd{u} (1-u^2) \\
		&= \frac{4\pi}{3} \sin{\theta'} \cos{\phi'}.
	\end{align*}
	Similarly we can complete the proof.
\end{proof}

\subsection{One tricky point on partial derivative below (5.108)}

Just below (5.108), it claims $\partial r / \partial z = \cos{\theta}$, this partial derivative treats $x$ and $y$ as constant. What is $\partial z / \partial r$? If we use expression $z = r \cos{\theta}$, then it is easy to go to result $\partial z / \partial r = \cos{\theta}$. But we have $\partial r / \partial z = \cos{\theta}$ already, how could $\partial z / \partial r = \partial r / \partial z$?
To resolve this ``paradox'', we should notice that when we do the partial derivative, we always need to specify what variables we keep as constant. In the first calculation $\partial r / \partial z = \cos{\theta}$, we keep $x, y$ as constant. However, in $\partial z / \partial r = \cos{\theta}$ we treat $\theta, \phi$ as constant. That's the reason why we have such inconsistent results.

\subsection{Derivation of (6.27) and (6.28)}

\begin{align*}
	\vb{J_l} &= \int \dd^3{x'} \delta(\vb{x} -\vb{x'}) \vb{J_l} \\
	&= - \frac{1}{4\pi} \int \dd^3{x'} \nabla'^2\qty(\frac{1}{\abs{\vb{x}-\vb{x'}}}) \vb{J_l} \\
	&= - \frac{1}{4\pi} \int \dd^3{x'} \grad' \cdot \qty[\grad'\qty(\frac{1}{\abs{\vb{x}-\vb{x'}}}) \vb{J_l}] + \frac{1}{4\pi} \int \dd^3{x'} \grad'\qty(\frac{1}{\abs{\vb{x}-\vb{x'}}}) (\grad' \cdot \vb{J_l}) \\
	&= - \frac{1}{4\pi} \int \dd^3{x'} \grad \qty(\frac{1}{\abs{\vb{x}-\vb{x'}}}) (\grad' \cdot \vb{J_l}) \\
	&= - \frac{1}{4\pi} \grad \int \dd^3{x'} \frac{\grad' \cdot \vb{J_l}}{\abs{\vb{x}-\vb{x'}}} \\
	&= - \frac{1}{4\pi} \grad \int \dd^3{x'} \frac{\grad' \cdot (\vb{J_l} + \vb{J_t})}{\abs{\vb{x}-\vb{x'}}} \\
	&= - \frac{1}{4\pi} \grad \int \dd^3{x'} \frac{\grad' \cdot \vb{J}}{\abs{\vb{x}-\vb{x'}}}.
\end{align*}

\begin{align*}
	\vb{J_t} &= \int \dd^3{x'} \delta(\vb{x} -\vb{x'}) \vb{J_t} \\
	&= - \frac{1}{4\pi} \int \dd^3{x'} \nabla^2\qty(\frac{1}{\abs{\vb{x}-\vb{x'}}}) \vb{J_t} \\
	&= - \frac{1}{4\pi} \int \dd^3{x'} \qty[- \grad(\div(\frac{\vb{J_t}}{\abs{\vb{x}-\vb{x'}}})) + \laplacian(\frac{\vb{J_t}}{\abs{\vb{x}-\vb{x'}}})] \\
	&= \frac{1}{4\pi} \int \dd^3{x'} \curl \curl(\frac{\vb{J_t}}{\abs{\vb{x}-\vb{x'}}}) \\
	&= \frac{1}{4\pi} \curl \curl \int \dd^3{x'} \frac{\vb{J}}{\abs{\vb{x}-\vb{x'}}}
\end{align*}

\subsection{Derivation of (9.37) in Jackson}

Basically we need to show the symmetric term can be related to quadrupole moment
\begin{equation}
	\int \dd[3]{x}(x_i J_j + x_j J_i) = -\int \dd[3]{x} x_i x_j \div{\vb{J}}.
\end{equation}
\begin{proof}
	We use $\delta_{ij} = \partial_i x_j$ to transform the symmetric term
	\begin{equation}
		x_i J_j + x_j J_i = \partial_k (x_i x_j J_k) - x_i x_j \partial_k J_k.
	\end{equation}
	The divergence term vanishes if the current distribution is localized.
\end{proof}

\subsection{Derivation of (9.46) in Jackson}
We have
\begin{align*}
	|(\vb{n} \times \vb{Q}) \times \vb{n}|^2 &= |\vb{Q}- (\vb{Q} \vdot \vb{n}) \vb{n}|^2\\
	&= \abs{\vb{Q}}^2 - 2\abs{\vb{Q} \vdot \vb{n}}^2 + \abs{\vb{Q} \vdot \vb{n}}^2 \\
	&= \abs{\vb{Q}}^2 - \abs{\vb{Q} \vdot \vb{n}}^2.
\end{align*}

\subsection{Angular momentum operator of wave mechanics (9.101) in Jackson}
The infinitesimal distance in spherical coordinates is
\begin{equation}
	\dd[2]{s} = \dd[2]{r} + r^2 \dd[2]{\theta} + r^2 \sin^2{\theta} \dd[2]{\phi},
\end{equation}
so the gradient operator in spherical coordinates is
\begin{equation}
	\grad = \vu{e}_r \pdv{r} + \vu{e}_{\theta} \frac{1}{r} \pdv{\theta} + \frac{1}{r \sin{\theta}} \vu{e}_{\phi},
\end{equation}
where $\vu{e}_r = (\sin{\theta}\cos{\phi}, \sin{\theta}\sin{\phi}, \cos{\theta})$, $\vu{e}_{\theta} = (\cos{\theta} \cos{\phi}, \cos{\theta} \sin{\phi}, - \sin{\theta})$,
and $\vu{e}_{\phi} = (-\sin{\phi}, \cos{\phi}, 0)$.
Hence we can write the angular momentum operator as
\begin{align*}
	\vb{L} &= \frac{1}{i} \vb{r} \times \grad, \\
	\vb{r} \times \grad &= (\vu{e}_r r) \times \qty(\vu{e}_r \pdv{r} + \vu{e}_{\theta} \frac{1}{r} \pdv{\theta} + \frac{1}{r \sin{\theta}} \vu{e}_{\phi}) \\
	&= \vu{e}_{\phi} \pdv{\theta} - \vu{e}_{\theta} \frac{1}{\sin{\theta}} \pdv{\phi}.
\end{align*}
And the differential operator $L^2$ can be obtained as
\begin{align*}
	L^2 &= -\qty(\vu{e}_{\phi} \pdv{\theta} - \vu{e}_{\theta} \frac{1}{\sin{\theta}} \pdv{\phi}) \vdot \qty(\vu{e}_{\phi} \pdv{\theta} - \vu{e}_{\theta} \frac{1}{\sin{\theta}} \pdv{\phi}) \\
	&= - \vu{e}_{\phi} \vdot \pdv{\theta} \qty(\vu{e}_{\phi} \pdv{\theta} - \vu{e}_{\theta} \frac{1}{\sin{\theta}} \pdv{\phi})
	+ \vu{e}_{\theta} \frac{1}{\sin{\theta}} \pdv{\phi} \vdot \qty(\vu{e}_{\phi} \pdv{\theta} - \vu{e}_{\theta} \frac{1}{\sin{\theta}} \pdv{\phi}) \\
	&= -\pdv[2]{\theta} - \cot{\theta} \pdv{\theta} -\frac{1}{\sin^2{\theta}} \pdv[2]{\phi},
\end{align*}
where we have used $\pdv*{\vu{e}_{\theta}}{\theta} = -\vu{e}_r$, $\pdv*{\vu{e}_{\phi}}{\theta} = 0$, $\pdv*{\vu{e}_{\theta}}{\phi} = \cos{\theta} \vu{e}_{\phi}$,
and $\pdv*{\vu{e}_{\phi}}{\phi} = -(\cos{\phi}, \sin{\phi}, 0)$.
Next let's look at $\vb{L}_{x,y,z}$
\begin{align*}
	\vb{r} \times \grad &= (-\sin{\phi} \vu{i} + \cos{\phi} \vu{j}) \pdv{\theta} \\
	&- (\cos{\theta} \cos{\phi} \vu{i} + \cos{\theta} \sin{\phi} \vu{j} - \sin{\theta} \vu{k}) \frac{1}{\sin{\theta}} \pdv{\phi}, \\
	L_x &= \frac{1}{i}\qty(-\sin{\phi}\pdv{\theta} - \cot{\theta}\cos{\phi}\pdv{\phi}), \\
	L_y &= \frac{1}{i}\qty(\cos{\phi}\pdv{\theta} - \cot{\theta}\sin{\phi}\pdv{\phi}), \\
	L_z &= -i\pdv{\phi}.
\end{align*}
Therefore
\begin{align*}
	L_+ &= L_x + iL_y = \e^{i\phi}\qty(\pdv{\theta} + i \cot{\theta} \pdv{\phi}), \\
	L_- &= L_x - iL_y = \e^{-i\phi}\qty(-\pdv{\theta} + i \cot{\theta} \pdv{\phi}).
\end{align*}

\subsection{Remark of delta function in page 120, Jackson}

To express $\delta(\vb{x} - \vb{x'}) = \delta(x_1 - x_1')\delta(x_2 - x_2')\delta(x_3 - x_3')$ in term of the coordinates $(\xi_1, \xi_2, \xi_3)$, related to $(x_1,x_2,x_3)$ via the Jacobian $J(x_i,\xi_i)$, we note that the meaningful quantity is $\delta(\vb{x} - \vb{x'}) \dd[3]{x}$.
Hence
\begin{equation}
	\delta(\vb{x} - \vb{x'}) = \frac{1}{\abs{J(x_i,\xi_i)}} \delta(\xi_1 - \xi_1') \delta(\xi_2 - \xi_2') \delta(\xi_3 - \xi_3')
\end{equation}
See problem 1.2.

\subsection{Derivation of (9.120) and (9.121) in Jackson}

Introduce the normalized form of spherical harmonics
\begin{equation}
	\vb{X}_{lm}(\theta, \phi) = \frac{1}{\sqrt{l(l+1)}} \vb{L} Y_{lm}(\theta, \phi).
\end{equation}
Using the completeness in a unit sphere
\begin{equation}
	\mathbb{I} = \int \dd{\Omega} \dyad{\vu{r}},
\end{equation}
we can calculate the orthogonality of vector shperical harmonics
\begin{align*}
	\int \dd{\Omega} (\vb{L}Y_{lm})^* \vdot (\vb{L}Y_{l'm'})
	&= \mel{lm}{\vb{L}\vdot\vb{L}}{l'm'} \\
	&= \int \dd{\Omega} \braket{lm}{\vu{r}} \ev{\vb{L}\vdot\vb{L}}{\vu{r}} \braket{\vu{r}}{l'm'} \\
	&= l(l+1) \int \dd{\Omega} Y_{lm}^* (\theta, \phi) Y_{l'm'} (\theta, \phi) \\
	&= l(l+1) \delta_{l l'}\delta_{m m'},
\end{align*}

To prove (9.121), we need to show $\vb{L}\vdot(\vb{r} \cross \vb{L}) = 0$.
\begin{align*}
	\vb{L}\vdot(\vb{r} \cross \vb{L}) &= \epsilon_{ijk} L_i x_j L_k \\
	&= \epsilon_{ijk} (x_j L_i + [L_i, x_j]) L_k \\
	&= i \epsilon_{ijk} \epsilon_{ijl} x_l L_k \\
	&= 2i \delta_{kl} x_l L_k \\
	&= 2i x_k L_k \\
	&= \frac{2i}{\hbar} \epsilon_{ijk} x_k x_i p_j = 0,
\end{align*}
where we used the commutator
\begin{equation}
	[L_i, x_j] = \frac{\epsilon_{ikl}}{\hbar} x_k [p_l, x_j] = -i \epsilon_{ikl} x_k \delta_{lj} = i \epsilon_{ijk} x_k,
\end{equation}
and the property of Levi-Civita symbol $\epsilon_{ijk} \epsilon_{ijl} = 2 \delta_{kl}$.




\end{document}
