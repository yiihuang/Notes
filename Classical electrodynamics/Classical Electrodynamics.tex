\documentclass[10pt]{article}
\pdfoutput=1
%\usepackage{NotesTeX,lipsum}
\usepackage{NotesTeX,lipsum}
%\usepackage{showframe}

\title{\begin{center}{\Huge \textit{Notes}}\\{{\itshape Classical Electrodynamics}}\end{center}}
\author{Yi Huang}


\affiliation{
University of Minnesota
}

\emailAdd{yihphysics@gmail.com}

\begin{document}
	\maketitle
	\flushbottom
	\newpage
	\pagestyle{fancynotes}
	\part{Caprice}
	\section{Spring 2019}\label{sec:spring2019}
	\begin{margintable}\vspace{.8in}\footnotesize
		\begin{tabularx}{\marginparwidth}{|X}
		Section~\ref{sec:spring 2019}. spring 2019\\
		\end{tabularx}
	\end{margintable}

\subsection{The angular integral in (4.16')}

Show the following equation is true
\begin{equation}
	\int \dd{\Omega} \vb{n} \cos{\gamma} = \frac{4 \pi \vb{n}'}{3},
\end{equation}
where $\vb{n} = \vb{i} \sin{\theta} \cos{\phi} + \vb{j} \sin{\theta} \sin{\phi} + \vb{k} \cos{\theta}$, and $\cos{\gamma} = \cos{\theta} \cos{\theta'} + \sin{\theta} \sin{\theta'} \cos(\phi - \phi')$.
\begin{proof}
	First look at the $x$ component of the integral:
	\begin{align*}
		\int \dd{\Omega} \sin{\theta} \cos{\phi} \cos{\gamma} &= \int \dd{\cos{\theta}} \dd{\phi} \sin{\theta} \cos{\phi} [cos{\theta} \cos{\theta'} + \sin{\theta} \sin{\theta'} \cos(\phi - \phi')] \\
		(\text{integrate over $\phi$}) &= \pi \int \dd{\cos{\theta}} \sin^2{\theta} \sin{\theta'} \cos{\phi'} \\
		(u = \cos{\theta}) &= \pi \sin{\theta'} \cos{\phi'} \int_{-1}^{1} \dd{u} (1-u^2) \\
		&= \frac{4\pi}{3} \sin{\theta'} \cos{\phi'}.
	\end{align*}
	Similarly we can complete the proof.
\end{proof}

\subsection{One tricky point on partial derivative below (5.108)}

Just below (5.108), it claims $\partial r / \partial z = \cos{\theta}$, this partial derivative treats $x$ and $y$ as constant. What is $\partial z / \partial r$? If we use expression $z = r \cos{\theta}$, then it is easy to go to result $\partial z / \partial r = \cos{\theta}$. But we have $\partial r / \partial z = \cos{\theta}$ already, how could $\partial z / \partial r = \partial r / \partial z$?
To resolve this ``paradox'', we should notice that when we do the partial derivative, we always need to specify what variables we keep as constant. In the first calculation $\partial r / \partial z = \cos{\theta}$, we keep $x, y$ as constant. However, in $\partial z / \partial r = \cos{\theta}$ we treat $\theta, \phi$ as constant. That's the reason why we have such inconsistent results.

\subsection{Derivation of (6.27) and (6.28)}

\begin{align*}
	\vb{J_l} &= \int \dd^3{x'} \delta(\vb{x} -\vb{x'}) \vb{J_l} \\
	&= - \frac{1}{4\pi} \int \dd^3{x'} \nabla'^2\qty(\frac{1}{\abs{\vb{x}-\vb{x'}}}) \vb{J_l} \\
	&= - \frac{1}{4\pi} \int \dd^3{x'} \grad' \cdot \qty[\grad'\qty(\frac{1}{\abs{\vb{x}-\vb{x'}}}) \vb{J_l}] + \frac{1}{4\pi} \int \dd^3{x'} \grad'\qty(\frac{1}{\abs{\vb{x}-\vb{x'}}}) (\grad' \cdot \vb{J_l}) \\
	&= - \frac{1}{4\pi} \int \dd^3{x'} \grad \qty(\frac{1}{\abs{\vb{x}-\vb{x'}}}) (\grad' \cdot \vb{J_l}) \\
	&= - \frac{1}{4\pi} \grad \int \dd^3{x'} \frac{\grad' \cdot \vb{J_l}}{\abs{\vb{x}-\vb{x'}}} \\
	&= - \frac{1}{4\pi} \grad \int \dd^3{x'} \frac{\grad' \cdot (\vb{J_l} + \vb{J_t})}{\abs{\vb{x}-\vb{x'}}} \\
	&= - \frac{1}{4\pi} \grad \int \dd^3{x'} \frac{\grad' \cdot \vb{J}}{\abs{\vb{x}-\vb{x'}}}.
\end{align*}

\begin{align*}
	\vb{J_t} &= \int \dd^3{x'} \delta(\vb{x} -\vb{x'}) \vb{J_t} \\
	&= - \frac{1}{4\pi} \int \dd^3{x'} \nabla^2\qty(\frac{1}{\abs{\vb{x}-\vb{x'}}}) \vb{J_t} \\
	&= - \frac{1}{4\pi} \int \dd^3{x'} \qty[- \grad(\div(\frac{\vb{J_t}}{\abs{\vb{x}-\vb{x'}}})) + \laplacian(\frac{\vb{J_t}}{\abs{\vb{x}-\vb{x'}}})] \\
	&= \frac{1}{4\pi} \int \dd^3{x'} \curl \curl(\frac{\vb{J_t}}{\abs{\vb{x}-\vb{x'}}}) \\
	&= \frac{1}{4\pi} \curl \curl \int \dd^3{x'} \frac{\vb{J}}{\abs{\vb{x}-\vb{x'}}}
\end{align*}


\end{document}
