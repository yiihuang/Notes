\documentclass[10pt]{article}
\pdfoutput=1
%\usepackage{NotesTeX,lipsum}
\usepackage{NotesTeX,lipsum}
%\usepackage{showframe}

\title{\begin{center}{\Huge \textit{Notes}}\\{{\itshape Classical Electrodynamics}}\end{center}}
\author{Yi Huang}


\affiliation{
University of Minnesota
}

\emailAdd{yihphysics@gmail.com}

\begin{document}
	\maketitle
	\flushbottom
	\newpage
	\pagestyle{fancynotes}
	\part{Classical Electrodynamics}
	\section{Spring 2019}\label{sec:spring2019}
	\begin{margintable}\vspace{.8in}\footnotesize
		\begin{tabularx}{\marginparwidth}{|X}
		Section~\ref{sec:spring2019}. spring 2019\\
		\end{tabularx}
	\end{margintable}

\subsection{The angular integral in (4.16')}

Show the following equation is true
\begin{equation}
	\int \dd{\Omega} \vb{n} \cos{\gamma} = \frac{4 \pi \vb{n}'}{3},
\end{equation}
where $\vb{n} = \vb{i} \sin{\theta} \cos{\phi} + \vb{j} \sin{\theta} \sin{\phi} + \vb{k} \cos{\theta}$, and $\cos{\gamma} = \cos{\theta} \cos{\theta'} + \sin{\theta} \sin{\theta'} \cos(\phi - \phi')$.
\begin{proof}
	First look at the $x$ component of the integral:
	\begin{align*}
		\int \dd{\Omega} \sin{\theta} \cos{\phi} \cos{\gamma} &= \int \dd{\cos{\theta}} \dd{\phi} \sin{\theta} \cos{\phi} [cos{\theta} \cos{\theta'} + \sin{\theta} \sin{\theta'} \cos(\phi - \phi')] \\
		(\text{integrate over $\phi$}) &= \pi \int \dd{\cos{\theta}} \sin^2{\theta} \sin{\theta'} \cos{\phi'} \\
		(u = \cos{\theta}) &= \pi \sin{\theta'} \cos{\phi'} \int_{-1}^{1} \dd{u} (1-u^2) \\
		&= \frac{4\pi}{3} \sin{\theta'} \cos{\phi'}.
	\end{align*}
	Similarly we can complete the proof.
\end{proof}

\subsection{One tricky point on partial derivative below (5.108)}

Just below (5.108), it claims $\partial r / \partial z = \cos{\theta}$, this partial derivative treats $x$ and $y$ as constant. What is $\partial z / \partial r$? If we use expression $z = r \cos{\theta}$, then it is easy to go to result $\partial z / \partial r = \cos{\theta}$. But we have $\partial r / \partial z = \cos{\theta}$ already, how could $\partial z / \partial r = \partial r / \partial z$?
To resolve this ``paradox'', we should notice that when we do the partial derivative, we always need to specify what variables we keep as constant. In the first calculation $\partial r / \partial z = \cos{\theta}$, we keep $x, y$ as constant. However, in $\partial z / \partial r = \cos{\theta}$ we treat $\theta, \phi$ as constant. That's the reason why we have such inconsistent results.

\subsection{Derivation of (6.27) and (6.28)}
Show that the current density (or any vector field) can be written as the sum of two terms,
\begin{equation}
	\vb{J} = \vb{J}_l + \vb{J}_t,
\end{equation}
where $\vb{J}_l$ is called the \textit{longitudinal} or \textit{irrotational} current and has $\curl{\vb{J}_l} = 0$, while $\vb{J}_t$ is called the \textit{transverse} or \textit{solenoidal} current and has $\div{\vb{J}_t} = 0$. Show that $\vb{J}_l$ and $\vb{J}_t$ can be constructed explicitly from $\vb{J}$ as follows:
\begin{align}
	\vb{J}_l &= -\frac{1}{4\pi} \grad \int \dd[3]{x'} \frac{\grad' \vdot {\vb{J}}}{\abs{\vb{x} - \vb{x'}}}, \\
	\vb{J}_t &= \frac{1}{4\pi} \curl \curl \int \dd[3]{x'} \frac{\vb{J}}{\abs{\vb{x}- \vb{x'}}}.
\end{align}
\begin{proof}
	It is obvious that $\curl{\vb{J}_l} = 0$ and $\div{\vb{J}_t} = 0$ are satisfied by the construction above. Next we only need to show $\vb{J} = \vb{J}_l + \vb{J}_t$. We start from the transverse component
	\begin{align*}
		\vb{J}_t &= \frac{1}{4\pi} \int \dd[3]{x'} \curl \curl \qty(\frac{\vb{J}(\vb{x'})}{\abs{\vb{x}- \vb{x'}}}) \\
		&= \frac{1}{4\pi} \int \dd[3]{x'} \grad \qty[\div(\frac{\vb{J}(\vb{x'})}{\abs{\vb{x}- \vb{x'}}})] - \frac{1}{4\pi} \int \dd[3]{x'} \laplacian \qty(\frac{\vb{J}(\vb{x'})}{\abs{\vb{x}- \vb{x'}}}) \\
		&= \frac{1}{4\pi} \int \dd[3]{x'} \grad \qty[\vb{J}(\vb{x'}) \vdot \grad(\frac{1}{\abs{\vb{x}- \vb{x'}}})] - \frac{1}{4\pi} \int \dd[3]{x'} \vb{J}(\vb{x'}) \qty(\laplacian \frac{1}{\abs{\vb{x}- \vb{x'}}}) \\
		&= - \frac{1}{4\pi} \grad \int \dd[3]{x'}  \vb{J}(\vb{x'}) \vdot \grad'\qty(\frac{1}{\abs{\vb{x}- \vb{x'}}}) + \frac{1}{4\pi} \int \dd[3]{x'} \vb{J}(\vb{x'}) 4\pi \delta(\vb{x} - \vb{x'}) \\
		&= \vb{J}(\vb{x}) - \frac{1}{4\pi} \grad \int \dd[3]{x'} \qty[\grad' \vdot \qty(\frac{\vb{J}(\vb{x'})}{\abs{\vb{x}- \vb{x'}}}) - \frac{\grad' \vdot \vb{J}(\vb{x'})}{\abs{\vb{x}- \vb{x'}}}] \\
		&= \vb{J} + \frac{1}{4\pi} \grad \int \dd[3]{x'} \frac{\grad' \vdot {\vb{J}}}{\abs{\vb{x} - \vb{x'}}} \\
		&= \vb{J} - \vb{J}_l.
	\end{align*}
	Hence
	\begin{equation}
		\vb{J}_t + \vb{J}_l = \vb{J}.
	\end{equation}

\end{proof}
An alternative proof.
\begin{align*}
	\vb{J_l} &= \int \dd^3{x'} \delta(\vb{x} -\vb{x'}) \vb{J_l} \\
	&= - \frac{1}{4\pi} \int \dd^3{x'} \nabla'^2\qty(\frac{1}{\abs{\vb{x}-\vb{x'}}}) \vb{J_l} \\
	&= - \frac{1}{4\pi} \int \dd^3{x'} \grad' \cdot \qty[\grad'\qty(\frac{1}{\abs{\vb{x}-\vb{x'}}}) \vb{J_l}] + \frac{1}{4\pi} \int \dd^3{x'} \grad'\qty(\frac{1}{\abs{\vb{x}-\vb{x'}}}) (\grad' \cdot \vb{J_l}) \\
	&= - \frac{1}{4\pi} \int \dd^3{x'} \grad \qty(\frac{1}{\abs{\vb{x}-\vb{x'}}}) (\grad' \cdot \vb{J_l}) \\
	&= - \frac{1}{4\pi} \grad \int \dd^3{x'} \frac{\grad' \cdot \vb{J_l}}{\abs{\vb{x}-\vb{x'}}} \\
	&= - \frac{1}{4\pi} \grad \int \dd^3{x'} \frac{\grad' \cdot (\vb{J_l} + \vb{J_t})}{\abs{\vb{x}-\vb{x'}}} \\
	&= - \frac{1}{4\pi} \grad \int \dd^3{x'} \frac{\grad' \cdot \vb{J}}{\abs{\vb{x}-\vb{x'}}}.
\end{align*}

\begin{align*}
	\vb{J_t} &= \int \dd^3{x'} \delta(\vb{x} -\vb{x'}) \vb{J_t} \\
	&= - \frac{1}{4\pi} \int \dd^3{x'} \nabla^2\qty(\frac{1}{\abs{\vb{x}-\vb{x'}}}) \vb{J_t} \\
	&= - \frac{1}{4\pi} \int \dd^3{x'} \qty[- \grad(\div(\frac{\vb{J_t}}{\abs{\vb{x}-\vb{x'}}})) + \laplacian(\frac{\vb{J_t}}{\abs{\vb{x}-\vb{x'}}})] \\
	&= \frac{1}{4\pi} \int \dd^3{x'} \curl \curl(\frac{\vb{J_t}}{\abs{\vb{x}-\vb{x'}}}) \\
	&= \frac{1}{4\pi} \curl \curl \int \dd^3{x'} \frac{\vb{J}}{\abs{\vb{x}-\vb{x'}}}
\end{align*}

\subsection{Derivation of (7.43) in Jackson}
For polarization parallel to the plane of incidence there is an angle of incidence for which there is no reflected wave. Assuming $\mu = \mu'$, show that the Brewster angle is
\begin{equation}
	i_B = \tan^{-1} \qty(\frac{n'}{n}).
\end{equation}
\begin{proof}
	From (7.41), in order that the reflected wave vanishes, we require
	\begin{equation}
		n'^2 \cos i - n \sqrt{n'^2 - n^2 \sin^2 i}.
	\end{equation}
	This is equivalent to
	\begin{equation}
		\qty(\frac{n'}{n})^2 \cos^2 i + \qty(\frac{n}{n'})^2 \sin^2 i = 1. \label{eq: 1.4.1}
	\end{equation}
	By some well-known trigonometric identities
	\begin{equation}
		\sin^2 i = \frac{\tan^2 i}{1+ \tan^2 i}\qc \cos^2 i = \frac{1}{1 + \tan^2 i},
	\end{equation}
	we may convert \eqref{eq: 1.4.1} into a quadratic equation of $\tan i$, and we can solve for it as
	\begin{equation}
		\tan i = \frac{n'}{n}.
	\end{equation}
	This is the expression of Brewster's angle.
\end{proof}

\subsection{Derivation of (9.37) in Jackson}

Basically we need to show the symmetric term can be related to quadrupole moment
\begin{equation}
	\int \dd[3]{x}(x_i J_j + x_j J_i) = -\int \dd[3]{x} x_i x_j \div{\vb{J}}.
\end{equation}
\begin{proof}
	We use $\delta_{ij} = \partial_i x_j$ to transform the symmetric term
	\begin{equation}
		x_i J_j + x_j J_i = \partial_k (x_i x_j J_k) - x_i x_j \partial_k J_k.
	\end{equation}
	The divergence term vanishes if the current distribution is localized.
\end{proof}

\subsection{Derivation of (9.46) in Jackson}
We have
\begin{align*}
	|(\vb{n} \times \vb{Q}) \times \vb{n}|^2 &= |\vb{Q}- (\vb{Q} \vdot \vb{n}) \vb{n}|^2\\
	&= \abs{\vb{Q}}^2 - 2\abs{\vb{Q} \vdot \vb{n}}^2 + \abs{\vb{Q} \vdot \vb{n}}^2 \\
	&= \abs{\vb{Q}}^2 - \abs{\vb{Q} \vdot \vb{n}}^2.
\end{align*}

\subsection{Angular momentum operator of wave mechanics (9.101) in Jackson}
The infinitesimal distance in spherical coordinates is
\begin{equation}
	\dd[2]{s} = \dd[2]{r} + r^2 \dd[2]{\theta} + r^2 \sin^2{\theta} \dd[2]{\phi},
\end{equation}
so the gradient operator in spherical coordinates is
\begin{equation}
	\grad = \vu{e}_r \pdv{r} + \vu{e}_{\theta} \frac{1}{r} \pdv{\theta} + \frac{1}{r \sin{\theta}} \vu{e}_{\phi},
\end{equation}
where $\vu{e}_r = (\sin{\theta}\cos{\phi}, \sin{\theta}\sin{\phi}, \cos{\theta})$, $\vu{e}_{\theta} = (\cos{\theta} \cos{\phi}, \cos{\theta} \sin{\phi}, - \sin{\theta})$,
and $\vu{e}_{\phi} = (-\sin{\phi}, \cos{\phi}, 0)$.
Hence we can write the angular momentum operator as
\begin{align*}
	\vb{L} &= \frac{1}{i} \vb{r} \times \grad, \\
	\vb{r} \times \grad &= (\vu{e}_r r) \times \qty(\vu{e}_r \pdv{r} + \vu{e}_{\theta} \frac{1}{r} \pdv{\theta} + \vu{e}_{\phi} \frac{1}{r \sin{\theta}} \pdv{\phi}) \\
	&= \vu{e}_{\phi} \pdv{\theta} - \vu{e}_{\theta} \frac{1}{\sin{\theta}} \pdv{\phi}.
\end{align*}
And the differential operator $L^2$ can be obtained as
\begin{align*}
	L^2 &= -\qty(\vu{e}_{\phi} \pdv{\theta} - \vu{e}_{\theta} \frac{1}{\sin{\theta}} \pdv{\phi}) \vdot \qty(\vu{e}_{\phi} \pdv{\theta} - \vu{e}_{\theta} \frac{1}{\sin{\theta}} \pdv{\phi}) \\
	&= - \vu{e}_{\phi} \vdot \pdv{\theta} \qty(\vu{e}_{\phi} \pdv{\theta} - \vu{e}_{\theta} \frac{1}{\sin{\theta}} \pdv{\phi})
	+ \vu{e}_{\theta} \frac{1}{\sin{\theta}} \pdv{\phi} \vdot \qty(\vu{e}_{\phi} \pdv{\theta} - \vu{e}_{\theta} \frac{1}{\sin{\theta}} \pdv{\phi}) \\
	&= -\pdv[2]{\theta} - \cot{\theta} \pdv{\theta} -\frac{1}{\sin^2{\theta}} \pdv[2]{\phi},
\end{align*}
where we have used $\pdv*{\vu{e}_{\theta}}{\theta} = -\vu{e}_r$, $\pdv*{\vu{e}_{\phi}}{\theta} = 0$, $\pdv*{\vu{e}_{\theta}}{\phi} = \cos{\theta} \vu{e}_{\phi}$,
and $\pdv*{\vu{e}_{\phi}}{\phi} = -(\cos{\phi}, \sin{\phi}, 0)$.
Next let's look at $\vb{L}_{x,y,z}$
\begin{align*}
	\vb{r} \times \grad &= (-\sin{\phi} \vu{i} + \cos{\phi} \vu{j}) \pdv{\theta} \\
	&- (\cos{\theta} \cos{\phi} \vu{i} + \cos{\theta} \sin{\phi} \vu{j} - \sin{\theta} \vu{k}) \frac{1}{\sin{\theta}} \pdv{\phi}, \\
	L_x &= \frac{1}{i}\qty(-\sin{\phi}\pdv{\theta} - \cot{\theta}\cos{\phi}\pdv{\phi}), \\
	L_y &= \frac{1}{i}\qty(\cos{\phi}\pdv{\theta} - \cot{\theta}\sin{\phi}\pdv{\phi}), \\
	L_z &= -i\pdv{\phi}.
\end{align*}
Therefore
\begin{align*}
	L_+ &= L_x + iL_y = \e^{i\phi}\qty(\pdv{\theta} + i \cot{\theta} \pdv{\phi}), \\
	L_- &= L_x - iL_y = \e^{-i\phi}\qty(-\pdv{\theta} + i \cot{\theta} \pdv{\phi}).
\end{align*}

\subsection{Remark of delta function in page 120, Jackson}

To express $\delta(\vb{x} - \vb{x'}) = \delta(x_1 - x_1')\delta(x_2 - x_2')\delta(x_3 - x_3')$ in term of the coordinates $(\xi_1, \xi_2, \xi_3)$, related to $(x_1,x_2,x_3)$ via the Jacobian $J(x_i,\xi_i)$, we note that the meaningful quantity is $\delta(\vb{x} - \vb{x'}) \dd[3]{x}$.
Hence
\begin{equation}
	\delta(\vb{x} - \vb{x'}) = \frac{1}{\abs{J(x_i,\xi_i)}} \delta(\xi_1 - \xi_1') \delta(\xi_2 - \xi_2') \delta(\xi_3 - \xi_3')
\end{equation}
See problem 1.2.

\subsection{Derivation of (9.120) and (9.121) in Jackson}

Introduce the normalized form of spherical harmonics
\begin{equation}
	\vb{X}_{lm}(\theta, \phi) = \frac{1}{\sqrt{l(l+1)}} \vb{L} Y_{lm}(\theta, \phi).
\end{equation}
Using the completeness in a unit sphere
\begin{equation}
	\mathbb{I} = \int \dd{\Omega} \dyad{\vu{r}},
\end{equation}
we can calculate the orthogonality of vector shperical harmonics
\begin{align*}
	\int \dd{\Omega} (\vb{L}Y_{lm})^* \vdot (\vb{L}Y_{l'm'})
	&= \mel{lm}{\vb{L}\vdot\vb{L}}{l'm'} \\
	&= \int \dd{\Omega} \braket{lm}{\vu{r}} \ev{\vb{L}\vdot\vb{L}}{\vu{r}} \braket{\vu{r}}{l'm'} \\
	&= l(l+1) \int \dd{\Omega} Y_{lm}^* (\theta, \phi) Y_{l'm'} (\theta, \phi) \\
	&= l(l+1) \delta_{l l'}\delta_{m m'},
\end{align*}

To prove (9.121), we need to show $\vb{L}\vdot(\vb{r} \cross \vb{L}) = 0$.
\begin{align*}
	\vb{L}\vdot(\vb{r} \cross \vb{L}) &= \epsilon_{ijk} L_i x_j L_k \\
	&= \epsilon_{ijk} (x_j L_i + [L_i, x_j]) L_k \\
	&= i \epsilon_{ijk} \epsilon_{ijl} x_l L_k \\
	&= 2i \delta_{kl} x_l L_k \\
	&= 2i x_k L_k \\
	&= \frac{2i}{\hbar} \epsilon_{ijk} x_k x_i p_j = 0,
\end{align*}
where we used the commutator
\begin{equation}
	[L_i, x_j] = \frac{\epsilon_{ikl}}{\hbar} x_k [p_l, x_j] = -i \epsilon_{ikl} x_k \delta_{lj} = i \epsilon_{ijk} x_k,
\end{equation}
and the property of Levi-Civita symbol $\epsilon_{ijk} \epsilon_{ijl} = 2 \delta_{kl}$.

\subsection{Derivation of (10.4) in Jackson}
Given the incident fields
\begin{align*}
	\vb{E}_{\mathrm{inc}} &= \vu{\epsilon}_0 E_0 \e^{i k \vu{n}_0\vdot \vb{x}}, \\
	\vb{H}_{\mathrm{inc}} &= \vu{n}_0 \cross \vb{E}_{\mathrm{inc}} / Z_0,
\end{align*}
and the scattered (radiated) fields
\begin{align*}
	\vb{E}_{\mathrm{sc}} &= \frac{1}{4 \pi \epsilon} k^2 \frac{\e^{ikr}}{r} [(\vu{n} \cross \vb{p}) \cross \vu{n} - \vu{n} \cross \vb{m} / c], \\
	\vb{H}_{\mathrm{sc}} &= \vu{n} \cross \vb{E}_{\mathrm{sc}} / Z_0,
\end{align*}
we can calculate differential scattering cross section
\begin{equation}
	\frac{\dd{\sigma}}{\dd{\Omega}}(\vu{n}, \vu*{\epsilon}; \vu{n}_0, \vu*{\epsilon}_0) = \frac{r^2 \frac{1}{2Z_0} \abs{\vu*{\epsilon}^* \vdot \vb{E}_{\mathrm{sc}}}^2}{\frac{1}{2Z_0} \abs{\vu*{\epsilon}_0^* \vdot \vb{E}_{\mathrm{inc}}}^2}.
\end{equation}
Show that the differential cross section can be written as
\begin{equation}
	\frac{\dd{\sigma}}{\dd{\Omega}}(\vu{n}, \vu*{\epsilon}; \vu{n}_0, \vu*{\epsilon}_0) =
	\frac{k^4}{4\pi \epsilon_0 E_0}^2 \abs{\vu*{\epsilon}^* \vdot \vb{p} + (\vu{n} \cross \vu*{\epsilon}^*) \vdot \vb{m}/c}^2.
\end{equation}
\begin{proof}
	Notice that $\vu*{\epsilon}$ is perpendicular to $\vu{n}$, so
	\begin{align*}
		\vu*{\epsilon}^* \vdot [(\vu{n} \cross \vb{p}) \cross \vu{n}] &= \vu{n} \vdot [\vu*{\epsilon}^* \cross (\vu{n} \cross \vb{p})] \\
		&= \vu{n} \vdot [(\vu*{\epsilon}^* \vdot \vb{p}) \vu{n} - (\vu*{\epsilon}^* \vdot \vu{n}) \vb{p}] \\
		&= \vu*{\epsilon}^* \vdot \vb{p}.
	\end{align*}
	For the magnetic moment part we have
	\begin{equation}
		- \vu*{\epsilon}^* \vdot (\vu{n} \cross \vb{m}) = \vb{m} \vdot (\vu{n} \cross \vu*{\epsilon}^*).
	\end{equation}
\end{proof}

\subsection{Derivation of (10.7) in Jackson}
Without losing generality, we define $\phi = 0$ in the scattering plane spaned by the vectors $\vu{n}_0$ anf $\vu{n}$, and $z$ axis along $\vu{n}_0$ (see Figure 10.1). Then we can write the polarization vectors in cartesian coordinates
\begin{align*}
	\vu*{\epsilon}_0 &= (\cos{\phi}, \sin{\phi}, 0), \\
	\vu*{\epsilon}_{\parallel} &= (\cos{\theta}, 0, -\sin{\theta}), \\
	\vu*{\epsilon}_{\perp} &= (0, 1, 0).
\end{align*}
We then average over initial polarization $\phi \to (0, 2\pi)$
\begin{gather*}
	\int_0^{2\pi} \dd{\phi} \abs{\vu*{\epsilon}_0 \vdot \vu*{\epsilon}_{\parallel}}^2 = \cos^2{\theta} \int_0^{2\pi} \dd{\phi} \cos^2{\phi} = \frac{1}{2} \cos^2{\theta}, \\
	\int_0^{2\pi} \dd{\phi} \abs{\vu*{\epsilon}_0 \vdot \vu*{\epsilon}_{\perp}}^2 = \int_0^{2\pi} \dd{\phi} \sin^2{\phi} = \frac{1}{2}.
\end{gather*}

\subsection{Derivation of (10.34) in Jackson}

The total scattering cross section per molecule of the gas is
\begin{equation}
	\sigma \approx \frac{k^4}{6 \pi N^2} \abs{\epsilon_r - 1}^2 \approx \frac{2k^4}{3 \pi N^2} \abs{n - 1}^2,
\end{equation}
where we have used the index of refraction $n = \sqrt{\epsilon}_r$, and assuming $\abs{n-1} \ll 1$, so
\begin{equation}
	\abs{\epsilon_r - 1} = \abs{n^2 - 1} = \abs{(n+1)(n-1)} \approx 2\abs{n-1}.
\end{equation}

\subsection{Molecular polarizability $\gamma_{\mathrm{mol}}$}
The polarization vector $\vb{P}$ was defined in (4.28) as
\begin{equation}
	\vb{P} = N \ev{\vb{p}_{\mathrm{mol}}}
\end{equation}
where $N$ is the number of particles per volume, $\ev{\vb{p}_{\mathrm{mol}}}$ is the average dipole moment of the molecules. This dipole moment is approximately proportional to the electric field acting on the molecule. To exhibit this dependence on electric field we define the \textit{molecular polarizability} $\gamma_{\mathrm{mol}}$ as the ratio of the average molecular dipole moment to $\epsilon_0$ times the applied field at the molecule. Taking account of the internal field (4.63), this gives:
\begin{equation}
	\ev{\vb{p}_{\mathrm{mol}}} = \epsilon_0 \gamma_{\mathrm{mol}} (\vb{E} + \vb{E}_i)
\end{equation}
Notice that $\gamma_{\mathrm{mol}}$ has the dimension of volume.

Next we derive (10.36) in Jackson.

The variation in index of refraction $\delta \epsilon$ for the $j$th cell with volume $v$ is
\begin{equation}
	\delta \epsilon = \pdv{\epsilon}{N} \cdot \frac{\Delta N_j}{v}. \label{eq: 1.12.1}
\end{equation}
We emphasize again $N$ is number density with unit one over volume, but $\Delta N_j$ is the departure from the mean of the number of molecules in the $j$th cell and $\Delta N_j$ is dimensionless.

From the Clausius-Mossotti relation (4.70)
\begin{equation*}
	\gamma_{\mathrm{mol}} = \frac{3}{N} \qty(\frac{\epsilon - \epsilon_0}{\epsilon + 2 \epsilon_0}),
\end{equation*}
we can calculate $\pdv*{\epsilon}{N}$ by taking derivative with respect to $N$ for Clausius-Mossotti relation (where we assume $\gamma_{\mathrm{mol}}$ is a constant)
\begin{gather*}
	\gamma_{\mathrm{mol}} = 3 \pdv{N} \qty(\frac{\epsilon - \epsilon_0}{\epsilon + 2 \epsilon_0}) \Longrightarrow \pdv{\epsilon}{N} = \frac{(\epsilon_r - 1)(\epsilon_r + 2)}{3N}.
\end{gather*}
Substitute into \eqref{eq: 1.12.1} we derived (10.36) in Jackson.

\subsection{Derivation of the basic Kirchhoff integral formula}
The scalar field $\psi(\vb{x})$ satisfies the Helmholtz wave equation
\begin{equation}
	(\laplacian + k^2) \psi(\vb{x}) = 0,
\end{equation}
We introduce a Green function for the Helmholtz wave equation $G(\vb{x}, \vb{x'})$, defined by
\begin{equation}
	(\laplacian + k^2) G(\vb{x}, \vb{x'}) = - \delta(\vb{x}-\vb{x'})
\end{equation}
According to Green's theorem, we have
\begin{equation}
	\int_V \dd[3]{x'} (\phi \laplacian \psi - \psi \laplacian \phi) = \oint_S \dd{a'} \vu{n'} \vdot (\psi \grad'{\phi} - \phi \grad'{\psi}),
\end{equation}
where $\vb{n'}$ is an \textit{inwardly} directed normal to the surface $S$, so the right hand side may flip a sign compared to the usual Green's theorem. We put $\phi = G$ and $\psi = \psi$, then
\begin{equation}
	\psi(\vb{x}) = \oint_S \dd{a'} \vu{n'} \vdot (\psi \grad'{G} - \phi \grad'{G}). \label{eq: 1.13.1}
\end{equation}
Take $G$ to be the infinite-space Green function describing outgoing waves,
\begin{equation}
	G(\vb{x}, \vb{x'}) = \frac{\e^{ikR}}{4\pi R},
\end{equation}
where $\vb{R} = \vb{x} - \vb{x'}$. The gradient of $G$ can be calculated by
\begin{align*}
	\grad'{\frac{\e^{ikR}}{R}} &= -\frac{\e^{ikR}}{R^2} \grad'{R} + ik\frac{\e^{ikR}}{R} \grad'{R} \\
	&= \frac{\e^{ikR}}{R}\qty(-ik + \frac{1}{R}) \frac{\vb{R}}{R} \\
	&= - \frac{\e^{ikR}}{R}ik\qty(1 + \frac{i}{kR}) \frac{\vb{R}}{R},
\end{align*}
where we use $\grad'{R} = -\vb{R}/R$. Therefore \eqref{eq: 1.13.1} becomes
\begin{equation}
	\psi(\vb{x}) = -\frac{1}{4\pi} \oint_S \frac{\e^{ikR}}{R} \vu{n'} \vdot \qty[\grad'{\psi} + ik \qty(1 + \frac{i}{kR}) \frac{\vb{R}}{R} \psi].
\end{equation}
Notice that $\psi(\vb{x})$ satisfies radiation condition in the neighborhood of $S_2$
\begin{equation}
	\psi \to f(\theta, \phi) \frac{\e^{ikr}}{r}\qc \frac{1}{\psi} \pdv{\psi}{r} \to \qty(ik - \frac{1}{r}),
\end{equation}
so the integral over $S_2$ will be propotional to at least $1/r$ and vanishes if $S_2$ goes to infinity.

\subsection{Mathematical inconsistency in Kirchhoff approximation}

\subsection{Why the integral over $S_2$ vanishes? Derivation of (10.98) in Jackson}
We need to show the integral in $S_2$ vanishes in the radiation zone ($r \to \infty$). The field $\psi$ satisfies the radiation conditions,
\begin{equation}
	\psi \to f(\theta, \phi) \frac{\e^{ikr}}{r}\qc \frac{1}{\psi} \pdv{\psi}{r} \to \qty(ik - \frac{1}{r}).
\end{equation}
Also we have
\begin{align*}
	\vb{n'} \vdot \grad'\psi(\vb{x'}) &= -vu{e}_r \vdot \qty(\pdv{\psi}{r'} \vu{e}_r + \frac{1}{r'}\pdv{\psi}{\theta'} \vu{e}_{\theta} + \frac{1}{r' \sin{\theta'}} \pdv{\psi}{\phi'} \vu{e}_{\phi}) \\
	&= -\pdv{\psi}{r'} = -\qty(ik - \frac{1}{r'}) \psi \\
\end{align*}
where $\vu{n'}$ is an inwardly directed normal to the surface, $\psi(\vb{x'})$ is the scalar field evaluated at the surface (see equation (10.75) in Jackson).
If we the field points $\vb{x}$ that we are interested in satisfy $\abs{\vb{x}} \ll \abs{\vb{x'}}$,
i.e. $r \ll r'$ the integral at $S_2$ is taken at infinity,
then $\vb{R} = \vb{x} - \vb{x'} \to -\vb{x'}$, $1/R = 1/\abs{\vb{x} - \vb{x'}} \to 1/r' + \vu{n'}\vdot\vb{x}/r'^2$, $\vb{R}/R \to -\vu{e}_r = \vu{n'}$, and the integrad of (10.77) becomes
\begin{equation}
	\frac{\e^{ikr'}}{r'} \qty[\qty(-ik + \frac{1}{r'}) + \qty(ik - \frac{1}{r'} - \vu{n'}\vdot \frac{\vb{x}}{r'^2})] \psi \propto 1/r'^3.
\end{equation}
If we perform the integral at the whole sphere extended by $S_2$ the integral will vanishes at least as the inverse of the radius of the sphere as the radius $r'$ goes to infinity. There remains the integral over $S_1$.



\subsection{Derivation of (9.123) in Jackson}
There are two terms in $\int \dd{\Omega} Y_{lm}^* \vb{r} \vdot \vb{H}$, where $\vb{H}$ is given in (9.122). The first term is propotional to
\begin{equation}
	\vb{r} \vdot \vb{L} \propto \vb{r} \vdot (\vb{r} \cross \grad) = (\vb{r} \cross \vb{r}) \vdot \grad = 0.
\end{equation}
The second term is propotional to
\begin{align*}
	\int\dd{\Omega} Y_{lm}^* (-i) \vb{r} \vdot (\grad \cross \vb{L} Y_{l'm'}) &=  \int\dd{\Omega} Y_{lm}^* (-i) (\vb{r} \cross \grad) \vdot \vb{L} Y_{l'm'} \\
	&= \int\dd{\Omega} Y_{lm}^* L^2 Y_{l'm'} \\
	&= l(l+1) \delta_{ll'} \delta_{mm'}.
\end{align*}

\subsection{Electric and magnetic dipole fields and radiation}
This is the derivation for section 9.2 and 9.3 in Jackson.

In the far (radiation) zone where $d \ll \lambda \ll r$, keep the first term in the vector potential
\begin{equation}
	\vb{A}(\vb{x}) = \frac{\mu_0}{4\pi} \frac{\e^{ikr}}{r} \int \dd[3]{x'} \vb{J}(\vb{x'}). \label{eq: 1.16.1}
\end{equation}
Notice that
\begin{equation}
	J_i = \delta_{ij} J_j = (\partial_j x_i) J_j = \partial_j(x_i J_j) - x_i (\partial_j J_j),
\end{equation}
we can do integration by parts and throw away the surface term because the source is localized
\begin{equation}
	\int \dd[3]{x'} \vb{J}{\vb{x'}} = - \int \dd[3]{x'} \vb{x'} \grad' \vdot \vb{J}(\vb{x'}).
\end{equation}
Using the continuity equation and assume the harmonic time dependence $\e^{-i\omega t}$, we have
\begin{align*}
	\div{\vb{J}(\vb{x},t)} &= -\pdv{\rho(\vb{x},t)}{t} \\
	\div{\vb{J}(\vb{x})} \e^{-i\omega t} &= i\omega \rho(\vb{x}) \e^{-i\omega t} \\
	\div{\vb{J}(\vb{x})} &= i\omega \rho(\vb{x}).
\end{align*}
The vector potential in \eqref{eq: 1.16.1} becomes
\begin{align}
	\vb{A}(\vb{x}) &= \frac{\mu_0}{4\pi} \frac{\e^{ikr}}{r} (-i\omega) \int \dd[3]{x'} \vb{x'} \rho(\vb{x'}) \nonumber \\
	&= -\frac{i \mu_0 \omega}{4\pi} \vb{p} \frac{\e^{ikr}}{r}. \label{eq: 1.16.2}
\end{align}
where
\begin{equation}
	\vb{p} = \int \dd[3]{x'} \vb{x'} \rho(\vb{x'})
\end{equation}
is the electric dipole moment.
The electric dipole fields can be derived from
\begin{align*}
	\vb{H} &= \frac{1}{\mu_0}\curl{\vb{A}}, \\
	\mu_0\epsilon_0\pdv{\vb{E}}{t} &= \curl{\mu_0 \vb{H}} \\
	-i\omega \epsilon_0 \vb{E} &= \curl{\vb{H}} \\
	\vb{E} &= \frac{i Z_0}{k} \curl{\vb{H}},
\end{align*}
where $Z_0 = \sqrt{\mu_0/\epsilon_0}$ is the impedance of free space. Substitute \eqref{eq: 1.16.2} into the equations above
\begin{align*}
	\vb{H} &= \frac{-i\omega}{4\pi} \curl(\vb{p} \frac{\e^{ikr}}{r}) \\
	&= \frac{-i\omega}{4\pi} \grad(\frac{\e^{ikr}}{r}) \cross \vb{p} \\
	&= \frac{ck^2}{4\pi} (\vu{n} \cross \vb{p}) \frac{\e^{ikr}}{r} \qty(1- \frac{1}{ikr}),
\end{align*}
where we used
\begin{equation}
	\grad{\frac{\e^{ikr}}{r}} = ik \vu{n} \frac{\e^{ikr}}{r} - \frac{\e^{ikr}}{r^2} \vu{n} = ik \vu{n} \frac{\e^{ikr}}{r} \qty(1- \frac{1}{ikr}),
\end{equation}
and $\vu{n} = \grad{r}$ is a unit vector in the direction of $\vb{x}$. Also we have the electric field
\begin{equation}
	\vb{E} = \frac{1}{4\pi \epsilon_0} \qty{k^2 (\vu{n} \cross \vb{p}) \cross \vu{n} \frac{\e^{ikr}}{r} + [3 \vu{n} (\vu{n} \vdot \vb{p}) - \vb{p}] \qty(\frac{1}{r^3} - \frac{ik}{r^2}) \e^{ikr}}.
\end{equation}
In the radiation zone the fields take on the limiting forms,
\begin{align*}
	\vb{H} &= \frac{ck^2}{4\pi} (\vu{n} \cross \vb{p}) \frac{\e^{ikr}}{r}, \\
	\vb{E} &= Z_0 \vb{H} \cross \vu{n}.
\end{align*}
The time-averaged power radiated per unit solid angle by the oscillationg dipole moment $\vb{p}$ is
\begin{align*}
	\frac{\dd{P}}{\dd{\Omega}} &= \frac{1}{2} \Re{r^2 \vu{n} \vdot \vb{E} \cross \vb{H}^*} \\
	&= \frac{1}{2}Z_0 r^2 \vu{n} \vdot [(\vb{H} \cross \vu{n}) \cross \vb{H}^*] \\
	&= \frac{1}{2}Z_0 r^2 (\vb{H} \cross \vu{n}) \vdot (\vb{H}^* \cross \vu{n}) \\
	&= \frac{c^2 Z_0}{32 \pi^2} k^4 \abs{(\vu{n} \cross \vb{p}) \cross \vu{n}}^2 \\
	&= \frac{c^2 Z_0}{32 \pi^2} k^4 \abs{\vb{p}}^2 \sin^2{\theta},
\end{align*}
where the angle $\theta$ is measured from the direction of $\vb{p}$.

For the purpose of comparison we list the results for magnetic dipole fields
\begin{gather}
	\vb{H} = \frac{1}{4\pi} \qty{k^2 (\vu{n} \cross \vb{m}) \cross \vu{n} \frac{\e^{ikr}}{r} + [3 \vu{n} (\vu{n} \vdot \vb{m}) - \vb{m}] \qty(\frac{1}{r^3} - \frac{ik}{r^2}) \e^{ikr}}, \\
	\vb{E} = - \frac{Z_0 k^2}{4\pi} (\vu{n} \cross \vb{m}) \frac{\e^{ikr}}{r} \qty(1- \frac{1}{ikr}).
\end{gather}
The radiation power is
\begin{equation}
\frac{\dd{P}}{\dd{\Omega}} = \frac{Z_0}{32 \pi^2} k^4 \abs{(\vu{n} \cross \vb{m}) \cross \vu{n}}^2.
\end{equation}
All the arguments concerning the behavior of the fields in the near and far zones are the same as for the electric dipole source, with the interchange $\vb{E} \to Z_0 \vb{H}$, $Z_0 \vb{H} \to -\vb{E}$, $\vb{p} \to \vb{m}/c$. Similarly the radiation pattern and total power radiated are the same for the two kinds of dipole. The only difference in the radiation fields is in the polarization. For an electric dipole the electric vector lies in the plane defined by $\vb{n}$ and $\vb{p}$, while for a magnetic dipole it is perpendicular to the plane defined by $\vb{n}$ and $\vb{m}$.

\subsection{Rapidity}
Suppose we have two successive boost along $x$ direction, so we have the following Lorentz transformation between three reference frame $K$, $K'$, and $K''$,
\begin{equation}
	x' = \gamma_2(x'' + \beta_2 t'')\qc t' = \gamma_2(\beta_2 x'' + t''),
\end{equation}
and
\begin{align*}
	x &= \gamma_1 (x' + \beta_1 t') \\
	&= \gamma_1 \gamma_2 [(1+\beta_1\beta_2) x'' + (\beta_1 + \beta_2)t''].
\end{align*}
We can define a new $\gamma$
\begin{align*}
	\gamma &= \gamma_1 \gamma_2 (1+ \beta_1\beta_2) \\
	&= \frac{1}{\sqrt{1-\beta_1^2}} \frac{1}{\sqrt{1-\beta_1^2}} (1+ \beta_1\beta_2) \\
	&= \qty[\frac{1-(\beta_1^2+\beta_2^2) + \beta_1^2\beta_2^2}{(1+ \beta_1\beta_2)^2}]^{-1/2} \\
	&= \qty[\frac{(1+ \beta_1\beta_2)^2 - (\beta_1 + \beta_2)^2}{(1+ \beta_1\beta_2)^2}]^{-1/2} \\
	&= \frac{1}{\sqrt{1-\beta^2}},
\end{align*}
where
\begin{equation}
	\beta = \frac{\beta_1 + \beta_2}{1+ \beta_1\beta_2}. \label{eq: 1.18.1}
\end{equation}
If we define rapidity $w$ as $\tanh{w} \equiv \beta$, then \eqref{eq: 1.18.1} just tells us rapidity can add for successive boost along the same direction
\begin{gather}
	\tanh{\omega} = \frac{\tanh{\omega_1} + \tanh{\omega_2}}{1+ \tanh{\omega_1}\tanh{\omega_2}} = \tanh{\omega_1 + \omega_2}, \\
	\Longrightarrow \omega = \omega_1 + \omega_2.
\end{gather}
This result can also be obtained by direct matrix multiplication
\begin{equation}
	x = \Lambda(\omega_1) x' = \Lambda(\omega_1) \Lambda(\omega_2) x'' = \Lambda(\omega_1 + \omega_2),
\end{equation}
where
\begin{equation}
	\Lambda(\omega) =
	\begin{pmatrix}
		\cosh{\omega} & \sinh{\omega} &0 &0 \\
		\sinh{\omega} & \cosh{\omega} &0 &0 \\
		0 &0 &1 &0 \\
		0 &0 &0 &1
	\end{pmatrix}.
\end{equation}

\subsection{Proof of (11.19) in Jackson}
The key point is to separate the components parallel and perpendicular to the boost direction $\vb{\beta}$.
\begin{align*}
	x'^0 &= \gamma (x^0 - \vb{\beta} \vdot \vb{x}), \\
	\vb{x'} &= (- \gamma \beta x^0 + \gamma x_{\parallel}) \vu*{\beta} + \vb{x}_{\perp} \\
	&= -\gamma \vb*{\beta} x^0 + \gamma (\vu*{\beta} \vdot \vb{x}) \vu*{\beta} + (\vb{x} - (\vu*{\beta} \vdot \vb{x}) \vu*{\beta}) \\
	&= \vb{x} + \frac{\gamma - 1}{\beta^2} (\vb*{\beta} \vdot \vb{x}) \vb*{\beta} - \gamma \vb*{\beta} x^0.
\end{align*}

\subsection{Derivation of (11.158) in Jackson}
Notice that by construction of the time component of the the spin 4-vector, in a inertial fram where the particle has velocity $c \vb*{\beta}$, we have
\begin{equation}
	S^0 = \vb*{\beta} \vdot \vb{S}.
\end{equation}
Hence, in the rest frame of the electron, the time component $S'^0 = 0$, and the space component is given by
\begin{align*}
	\vb{S'} = \vb{s} &= \vb{S} + \frac{\gamma - 1}{\beta^2} (\vb*{\beta} \vdot \vb{S}) \vb*{\beta} + \gamma \vb*{\beta} S^0 \\
	&= \vb{S} + \frac{\gamma - 1}{\beta^2} (\vb*{\beta} \vdot \vb{S}) \vb*{\beta} + \gamma \vb*{\beta} (\vb*{\beta} \vdot \vb{S})\\
	&= \vb{S} + \frac{\gamma - 1 + \gamma \beta^2}{\beta^2} (\vb*{\beta} \vdot \vb{S}) \vb*{\beta} \\
	&= \vb{S} - \frac{\gamma}{\gamma + 1} (\vb*{\beta} \vdot \vb{S}) \vb*{\beta},
\end{align*}
where we use
\begin{align*}
	\frac{\gamma - 1 + \gamma \beta^2}{\beta^2} &= \frac{\gamma^{-1} -1}{\beta^2} \\
	&= \frac{\gamma^{-2} -1}{\beta^2 (\gamma^{-1} +1)} \\
	&= - \frac{1}{\gamma^{-1} + 1} \\
	&= - \frac{\gamma}{\gamma + 1}.
\end{align*}

\subsection{Derivation of (11.150) in Jackson}
For a general Lorentz transformation from $K$ to $K'$ moving with velocity $\vb{v}$ relative to $K$, the transformation of electromagnetic fields can be written as
\begin{align*}
	\vb{E'} &= \gamma(\vb{E} + \vb*{\beta} \cross \vb{B}) - \frac{\gamma^2}{\gamma + 1} \vb*{\beta} (\vb*{\beta} \vdot \vb{E}), \\
	\vb{B'} &= \gamma(\vb{B} - \vb*{\beta} \cross \vb{E}) - \frac{\gamma^2}{\gamma + 1} \vb*{\beta} (\vb*{\beta} \vdot \vb{B}).
\end{align*}
The inverse transformation is obtained by putting $\vb*{\beta} \to -\vb*{\beta}$
\begin{align*}
	\vb{E} &= \gamma(\vb{E'} - \vb*{\beta} \cross \vb{B'}) - \frac{\gamma^2}{\gamma + 1} \vb*{\beta} (\vb*{\beta} \vdot \vb{E'}), \\
	\vb{B} &= \gamma(\vb{B'} + \vb*{\beta} \cross \vb{E'}) - \frac{\gamma^2}{\gamma + 1} \vb*{\beta} (\vb*{\beta} \vdot \vb{B'}).
\end{align*}
If no magnetic field exists in a certain frame $K'$, as for example with one or more point charges at rest in $K'$, then in the frame $K$ the magnetic field $\vb{B}$ and the electric field $\vb{E}$ are linked by the simple relation
\begin{equation}
	\vb{B} = \vb*{\beta} \cross \vb{E}.
\end{equation}
\begin{proof}
	If $\vb{B'}=0$, then in the frame $K$
	\begin{align}
		\vb{B} &= \gamma \vb*{\beta} \cross \vb{E'}, \label{eq: 1.21.1}\\
		\vb{E} &= \gamma \vb{E'} - \frac{\gamma^2}{\gamma +1} \vb*{\beta} (\vb*{\beta} \cross \vb{E'}). \label{eq: 1.21.2}
	\end{align}
	Use $\vb*{\beta} \cross $ acting on \eqref{eq: 1.21.2} we have
	\begin{align*}
		\vb*{\beta} \cross \vb{E} &= \gamma \vb*{\beta} \cross\vb{E'} - \frac{\gamma^2}{\gamma +1} (\vb*{\beta} \cross\vb*{\beta}) (\vb*{\beta} \cross \vb{E'}) \\
		&= \gamma \vb*{\beta} \cross\vb{E'}.
	\end{align*}
	Pluging back to \eqref{eq: 1.21.1}, we have
	\begin{equation}
		\vb{B} = \gamma \vb*{\beta} \cross \vb{E'} = \vb*{\beta} \cross \vb{E}.
	\end{equation}

\end{proof}

\subsection{Relativistic Doppler shift, (11.30) in Jackson}
Consider a plane wave of frequency $\omega$ and wave vector $\vb{k}$ in the inertial frame $K$. In the moving frame $K'$ this wave will have, in general, a different frequency $\omega'$ and wave vector $\vb{k'}$, but the phase of the wave is an invariant:
\begin{equation}
	\phi = \omega t - \vb{k} \vdot \vb{x} = \omega' t' - \vb{k'} \vdot \vb{x'}.
\end{equation}
Define
\begin{equation}
	k^0 \equiv \omega / c\qc k'^0 \equiv \omega' / c,
\end{equation}
then we see that the invariance of phase becomes
\begin{equation}
	k^0 x^0 - \vb{k} \vdot \vb{x} = k'^0 x'^0 - \vb{k'} \vdot \vb{x'}.
\end{equation}
Hence we can define a 4-vector $k^{\mu} = (k^0, \vb{k})$, and the phase is a scalar product of $k^{\mu}$ and $x^{\mu}$:
\begin{equation}
	\phi = k^{\mu} x_{\mu},
\end{equation}
which is obviously Lorentz invariant. For light waves, we have $\abs{\vb{k}} = k^0$, so we obtain
\begin{equation}
	k^{\mu} k_{\mu} = (k^0)^2 - \abs{\vb{k}}^2 = 0.
\end{equation}
Suppose $\theta$ and $\theta'$ are angles of $\vb{k}$ and $\vb{k'}$ relative to the direction of $\vb{v}$. Then we obtain the familiar Doppler shift formulas
\begin{align*}
	\omega' &= \gamma \omega (1 - \beta \cos{\theta}), \\
	\tan{\theta'} &= \frac{\sin{\theta}}{\gamma (\cos{\theta - \beta})}.
\end{align*}
\subsection{Problem 11.19 in Jackson: Particle decay}
A particle of mass $M$ and 4-momentum $P$ decays into two particles of masses $m_1$ and $m_2$.
\begin{enumerate}[(a)]
	\item Use the conservation of energy and momentum in the form $p_2= P - p_1$, and the invariance of scalar products of 4-vectors to show that the total energy of the first particle in the rest frame of the decaying particle is 
		\begin{equation}
			E_1 = \frac{M^2 + m_1^2 - m_2^2}{2M}.
		\end{equation}
		and that $E_2$ is obtained by interchanging $m_1$ and $m_2$. 

	\begin{proof}
		Consider the scalar product of the momentum
		\begin{equation}
			p_2^2 = (P - p_1)^2 = P^2 + p_1^2 - 2P\vdot p_1,
		\end{equation}
		and use $P^2 = M^2$, $p_i^2 = m_i^2$, and notice that $P \vodt p_i$ is also a scalar, so let's consider the rest frame of the parent particle, such that $P = (M, \vb{0})$, $p_i = (E_i, \vb{p_i})$. Then we have 
		\begin{equation}
			P \vdot p_1 = 2M E_1.
		\end{equation}
		As a result we prove the assertion 
		\begin{equation}
			E_1 = \frac{M^2 + m_1^2 - m_2^2}{2M}.
		\end{equation}
	\end{proof}	
	\item Show that the kinetic energy $T_i$ of the $i$ th particle in the same frame is 
		\begin{equation}
			T_i = \Delta M \qty(1 - \frac{m_i}{M} - \frac{\Delta M}{2M}),
		\end{equation}
		where $\Delta M = M - m_1 - m_2$ is the mass excess or $Q$ value of the process.
		\begin{proof}
			The kinetic energy is 
			\begin{align*}
				T_1 &= E_1 - m_1 = \frac{M^2 + m_1^2 - m_2^2 - 2M m_1}{2M} \\
					&= \frac{(M-m_1-m_2)(M-m_1+m_2)}{2M} \\
					&= \Delta M \qty(1 - \frac{m_1}{M} - \frac{\Delta M}{2M}).
			\end{align*}
		\end{proof}
\end{enumerate}
\subsection{Problem 11.20 in Jackson}
Using conservation of momentum and energy and the invariance of scalar products of 4-vectors show that, if the opening angle $\theta$ between the two tracks is measured, the mass of the decaying particle can be found from the formula 
\begin{equation}
	M^2 = m_1^2 + m_2^2 + 2E_1 E_2 - 2p_1 p_2 \cos{\theta},
\end{equation}
where here $p_1$ and $p_2$ are the magnitudes of the 3-momenta.
\begin{proof}
	Use the conservation of 4-momentum
	\begin{equation}
		P = p_1 + p_2,
	\end{equation}
	we have the scalar product 
	\begin{align*}
		P^2 &= p_1^2 + p_2^2 + 2 p_1 \vdot p_2, \\
		M^2 &= m_1^2 + m_2^2 + 2 (E_1E_2 - \vb{p_1} \vdot \vb{p_2}).
	\end{align*}
\end{proof}
\subsection{Problem 11.21 in Jackson}
If a system of mass $M$ decays or transforms at rest into a number of particles, the sum of whose masses is less than $M$ by an amount $\Delat M$, show that the maximum kinetic energy of the $i$ th particle (mass $m_i$) is 
\begin{equation}
	(T_i)_{\max} = \Delta M \qty(1 - \frac{m_i}{M} - \frac{\Delta M}{2M}).
\end{equation}
\begin{proof}
	Notice that the kinetic energy $T_i$ is determined by the 3-momentum $\vb{p_i}$ via 
	\begin{equation}
		T_i = \sqrt{m_i^2 + \vb{p_i}^2} - m_i.
	\end{equation}
	Hence to get the maximum kinetic energy, we need to maximized $\abs{\vb{p_i}}$. Since the 3-momentum is conserved, $\abs{\vb{p_i}}$ get maximized if all other particles move to the opposite of $\vb{p_i}$. In this case, we may treat the decay products as equivalently two particles with mass $m_i$ and $(M - \Delta M - m_i)$. Recall our familiar result of decay into two particles, we have 
	\begin{equation}
		(T_i)_{\max} = \Delta M \qty(1 - \frac{m_i}{M} - \frac{\Delta M}{2M}).
	\end{equation}
\end{proof}
\subsection{Problem 11.23 in Jackson}
In a collision process a particle of mass $m_2$, at rest in the laboratory, is struck by a particle of mass $m_1$, momentum $\vb{p}_L$ and total energy $E_L$. In the collision the two inital particles are transformed into two others of masses $m_3$ and $m_4$.
\begin{enumerate}[(a)]
	\item Use invariant scalar products to show that the total energy $W$ in the center of momentum (cm) frame has its square given by 
		\begin{equation}
			W^2 = m_1^2 + m_2^2 + 2m_2 E_L,
		\end{equation}
		and that the cms 3-momentum $\vb{p}_0$ is 
		\begin{equation}
			\vb{p}_0 = \frac{m_2 \vb{p}_L}{W}.
		\end{equation}
		\begin{proof}
			We use subscript 0 to indicate quantities in cm frame. One property of cm frame is the total momentum is zero ($\vb{p}_{10} = -\vb{p}_{20} = \vb{p}_0$).
			\begin{equation}
				W^2 = (E_{10} + E_{20})^2 = (E_{10} + E_{20})^2 - (\vb{p}_{10} + \vb{p}_{20})^2 = (p_{1} + p_{2})^2,
			\end{equation}
			where $p_{i0}$ are the 4-momentum of $i$ th particle. Since the scalar product is invariant, we can evaluate it in the lab frame where $p_1 = (E_L, \vb{p}_L)$ and $p_2 = (m_2, \vb{0})$.
			\begin{equation}
				W^2 = (p_{1} + p_{2})^2 = p_1^2 + p_2^2 + 2p_1 \vdot p_2 = m_1^2 + m_2^2 + 2m_2 E_L. 
			\end{equation}
			To find $\vb{p}_0$, we make use of the invariance of $p_1 \vdot p_2$, so in the lab frame 
			\begin{equation}
				(p_1 \vdot p_2)^2 = (m_2 E_L)^2 = m_2^2 (p_L^2 + m_1^2).
			\end{equation}
			In the cm frame 
			\begin{align*}
				(p_1 \vdot p_2)^2 &= (E_{10}E_{20} + p_0^2)^2 \\
								  &= (\sqrt{(m_1^2 + p_0^2)(m_2^2 + p_0^2)} + p_0^2)^2 \\
								  &= p_0^2 (m_1^2 + m_2^2 + 2 p_0^2 + 2E_{10}E_{20}) + m_1^2 m_2^2 \\
								  &= p_0^2(E_{10}+E_{20})^2 + m_1^2m_2^2.
			\end{align*}
			Compare with the results we have 
			\begin{equation}
				m_2 p_L = p_0 W \Longrightarrow \vb{p}_0 = \frac{m_2 \vb{p}_L}{W}.
			\end{equation}
		\end{proof}
	\item Show that the Lorentz transformation parameters $\beta$ and $\gamma$ descibing the velocity of the cm frame in the laboratory are
		\begin{equation}
			\vb*{\beta} = \frac{\vb{p}_L}{m_2 + E_L}\qc \gamma = \frac{m_2 + E_L}{W}.
		\end{equation}
		\begin{proof}
			The total 4-momentum $P = p_1 + p_2$ in the rest frame and the cm frame are related by a boost characterized by $\gamma$ and $\beta$.
		 \begin{equation}
			 (E_L + m_2, \vb{p}_L) \; \text{boost} \Rightarrow (W, \vb{0}).
		\end{equation}
		So we have the Lorentz transformation 
		\begin{align*}
			E_L + m_2 &= \gamma W, \\
			p_L &= \gamma \beta W, 
		\end{align*}
		which are equivalent to 
		\begin{equation}
			\vb*{\beta} = \frac{\vb{p}_L}{m_2 + E_L}\qc \gamma = \frac{m_2 + E_L}{W}.
		\end{equation}
		\end{proof}
\end{enumerate}
\subsection{Derivation of (12.47) in Jackson}
In this section, they actually discuss two types of spatial variation of the magnetic field. The first type is to be considered as a gradient perpendicular to the direction of $\vb{B}$, and this corresponds to (12.47). In other words, the direction of magnetic field does not change for the first type of spatial variation:
\begin{equation}
	\vb{B}(\vb{x}) = B(\vb{x}) \vu{e}_B,
\end{equation}
where $\vu{e}_B$ is a constant unit vector.
Therefore, when we expand $\vb{B}$ at a point of interest (set this point as the origin), we obtain
\begin{equation}
	\vb{B}(\vb{x}) \approx \vb{B}(\vb{x} = 0) + (\vb{x} \vdot \grad') \eval{\vb{B}(\vb{x'})}_{\vb{x'} = 0}, \label{eq: 1.23.1}
\end{equation}
where the gradient of $\vb{B}$ can be calculated as follows:
\begin{equation}
	\grad{\vb{B}(\vb{x})} = \grad{[B(\vb{x}) \vu{e}_B]} = [\grad{B(\vb{x})}] \vu{e}_B .
\end{equation}
The gradient of the field is perpendicular to the field, which means
\begin{align}
	\grad{B(\vb{x})} &= \vu{n} \pdv{B}{\xi}, \\
	\vu{n} \vdot \vb{B} &= 0, \\
	\vb{B}(\vb{x} = 0) &= B_0 \vu{e}_B = \vb{B}_0.
\end{align}
Substitute into \eqref{eq: 1.23.1}, we have
\begin{equation}
	\vb{B}(\vb{x}) \approx \vb{B}_0 + (\vu{n} \vdot \vb{x}) \eval{\pdv{B}{\xi}}_0 \vu{e}_B = \vb{B}_0 \qty(1 + \frac{1}{B_0} \eval{\pdv{B}{\xi}}_0 (\vu{n} \vdot \vb{x})).
\end{equation}

\subsection{Derivation of (13.2) in Jackson}
$p$ and $p'$ are 4-momentum of the electron before and after the collision. Suppose the collision is elastic, i.e. $E = E'$, then we have
\begin{align*}
	Q^2 &= - (p - p')^2 \\
	&= -[(E - E')^2 - (\vb{p} - \vb{p'})^2] \\
	&= (\vb{p} - \vb{p'})^2 \\
	&= (2 \abs{\vb{p}} \sin{\theta/2})^2.
\end{align*}
where $\theta$ is the angle between $\vb{p}$ and $\vb{p'}$ in the rest frame of the heavy particle (incident particle), and $\abs{\vb{p}} = \abs{\vb{p'}}$ if $M \gg m$. Hence we have the differential relationship
\begin{equation}
	\dd{Q^2} = 2 \abs{\vb{p}}^2 \sin{\theta} \dd{\theta} = \frac{\abs{\vb{p}}^2}{\pi} \dd{\Omega},
\end{equation}
where we have used $\dd{\Omega} = 2\pi \sin{\theta} \dd{\theta}$ for azimuthal symmetric collision. Therefore we have
\begin{align*}
	\dv{\sigma}{\Omega} = \dv{\sigma}{Q^2} \dv{Q^2}{\Omega} = 4 \pi \qty(\frac{ze^2}{\beta c Q^2}).
\end{align*}

\subsection{Derivation of (13.3) in Jackson}
Energy loss by the incident particle is equal to the kinetic energy imparted to the initially stationary electron, because of the energy conservation. Suppose $p$ and $p'$ are four momentum of the electron before and after the collision. In the electron's rest frame (before the collision) we find
\begin{equation}
	p = (mc, \vb{0})\qc p' = (E'/c, \vb{p}),
\end{equation}
then we can calculate $Q^2$
\begin{align*}
	Q^2 &= -(p-p')^2\\
	&= \vb{p}^2 - (E'- mc^2)^2/c^2 \\
	&= \vb{p}^2 - T^2/c^2 \\
	&= (E'^2 - m^2c^4)/c^2 - T^2/c^2 \\
	&= [(T+mc^2)^2 - m^2c^4 - T^2]/c^2 \\
	&= 2mT,
\end{align*}
where by definition of kinetic energy $T = E' - mc^2$.

\subsection{Derivation of (13.4) in Jackson}
In the rest frame of the electron, we want to evaluate the maximum kinetic energy of the electron, which is kicked by the incident particle. We go to section 13 of \textit{Classical Field Theory} by Landau. There is a comprehensive discussion of elastic collision. We look at the equation (13.7), which is derived under the assumption that the second particle (i.e., the electron in our case) is at rest, which means we are in the rest frame of the second particle. By relating the subscript 1 with our incident particle with energy $E$ mass $M$, and subscript 2 with our electron with mass $m$, we find the maximum energy of electron after the collision (in nature units)
\begin{equation}
	E_{\max} = m \frac{(E + m)^2 + (E^2 - M^2)}{(E + m)^2 - (E^2 - M^2)},
\end{equation}
where we set $\theta_2 = 0$, i.e., the transferred momentum $\vb{p'}_2$ is along the same direction of the momentum of the first particle $\vb{p}_1$. Hence the maximum kinetic energy of the electron
\begin{align*}
	T_{\max} &= E_{\max} - m = m \frac{2(E^2 - M^2)}{M^2 + m^2 + 2mE} \\
	&= m \frac{2(\gamma^2 M^2 - M^2)}{M^2 + m^2 + 2mE} \\
	&= \frac{2\gamma^2 \beta^2 m }{1+m^2/M^2 + 2mE/M^2}.
\end{align*}
where we write the energy of the incident particle in the rest frame of electron as $E = \gamma M$, and use the identity $\gamma^2 - 1 = \gamma^2 \beta^2$.
At the condition of equal mass $m = M$, we have
\begin{equation}
	T_{\max} = \frac{2(\gamma^2 - 1)m}{2(1+\gamma )} = (\gamma - 1) m.
\end{equation}
which is consistent with the discussion following equation (13.4) in Jackson.

\subsection{Physical meaning of (13.6) in Jackson}
The energy loss per unit distance in collisions with energy transfer $T$ for a heavy particle passing through matter with $N$ atoms per unit volume, each with $Z$ electrons, is given by
\begin{equation}
	\dd{E} = \dd{n} T = NZ \dd{\sigma} \dd{x} T.
\end{equation}
where $\dd{n} = NZ \dd{\sigma} \dd{x}$ is the number of incident particles involving with collisions passing when passing through the matter along a distance $\dd{x}$. the differential cross section $\dd{\sigma}$ corresponds to the specific energy transfer $T$, such that
\begin{equation}
	\dd{\sigma} = \dv{\sigma}{T} \dd{T}.
\end{equation}
Eventually we have
\begin{equation}
	\dv{E}{x} = NZ \int_{\epsilon}^{T_{\max}} T \dv{\sigma}{T} \dd{T}.
\end{equation}


\subsection{Elementary charge in Gaussian unit}
The unit of charge in Gaussian unit is statcoulomb
\begin{equation}
	[charge] = \mathrm{cm}^{3/2} \mathrm{g}^{1/2} \mathrm{s}^{-1} = \mathrm{cm}^{1/2} \mathrm{erg}^{1/2}.
\end{equation}
Charge in Gaussian unit is of the same dimension as the square root of the product of length and energy $\sqrt{E \cdot L}$. Hence we may express it in unit $\sqrt{\mathrm{MeV} \cdot \mathrm{cm}}$. Notice that
\begin{equation}
	1 \mathrm{erg} = 624.15 \mathrm{GeV} = 6.2415 \times 10^{5} \mathrm{MeV},
\end{equation}
so we can express the elementary charge as
\begin{equation}
	e = 4.80320425(10)\times10^{-10} \mathrm{statcoulomb} = \sqrt{1.4399764 \times 10^{-13} \mathrm{MeV} \cdot \mathrm{cm}}.
\end{equation}

\subsection{Instantaneous power radiated by an accelerated charge is Lorentz invariant, section 14.2 in Jackson}
Radiated electromagnetic energy behaves like a zero component of a 4-vector
\begin{equation}
	P^{\mu} = \frac{1}{c} \int T^{\mu 0} \dd{V},
\end{equation}
where $T^{\mu \nu}$ is the energy-momentum tensor, see section 32 in Landau.

Now look at the instantaneous rest frame $K$ (co-moving inertial frame) of the charge particle at a specific moment $t$. The 4-momentum radiated during $t$ and $t + \dd{t}$ is
\begin{equation}
	\dd{P} = (\dd{E}/c, \dd{\vb{p}}),
\end{equation}
but $\dd{\vb{p}} = 0$ according to the definition of instantaneous rest frame. Suppose now we boost into another inertial frame $K'$, and $K'$ moves with velocity $\vb{V}$ relative to $K$, choose the coordinate system properly such that $\vb{V}$ is along $x^1$ axis of $K$ and $K'$.
\begin{gather}
	\dd{E'}/c = \gamma (\dd{E}/c + (V/c) \dd{p}^1) = \gamma \dd{E}/c, \\
	\dd{t'} = \gamma (\dd{t} + (V/c) \dd{x}^1) = \gamma \dd{t}.
\end{gather}
where $\gamma = 1/\sqrt{1-V^2/c^2}$, and $\dd{t}$ is actually the proper time of the charged particle at this instantaneous rest frame $K$. Hence we can calculate the power in frame $K'$
\begin{equation}
	\dv{E'}{t'} = \frac{\gamma}{\gamma} \dv{E}{t} = \dv{E}{t}.
\end{equation}
Thus, the instantaneous power radiated by the charge is invariant in all the inertial frame.



\part{Classical Field Theory}
This is based on \textit{Classical Field Theory} by Landau and Lifshitz.
\section{Spring 2019}

\subsection{Sum of velocity (5.2) in Landau}
We need to prove the sum of two velocities each smaller than the velocity of light is again not greater than the velocity of light. There are multiple ways of proof.
\begin{enumerate}
	\item Subtract $c$ by $v$, where $v$ is the result from the sum of $v'$ and $V$, and we have
	\begin{equation}
		c - v \propto (c-V)(1-v'/c) >0,
	\end{equation}
	where we use $V<c$ and $v'<c$ because of the finiteness of velocity.
	\item Equivalently we need to prove
	\begin{equation}
		\frac{a+b}{1+ab} <1\qif 0 \le a<1, 0 \le b<1.
	\end{equation}
	This is equivalent to the inequality
	\begin{equation}
		a(1-b) + b <1.
	\end{equation}
	Note that $0 \le a, b<1$ the left hand side is the parametric equation for an interval $[a,1)$.
	\item There is similarity between the velocity sum in relativity and the sum of two tangent hyperbolic functions
	\begin{equation}
		\tanh{(\psi_1 + \psi_2)} = \frac{\tanh{\psi_1} + \tanh{\psi_2}}{1 + \tanh{\psi_1} \tanh{\psi_2}}.
	\end{equation}
	Given that $a = \tanh{\psi_1}$, $b = \tanh{\psi_2}$, and $0 \le a,b <1$, we have $0 \le \psi_{1,2} < \infty$. Hence $0 \le \tanh{(\psi_1 + \psi_2)} <1$.
	\end{enumerate}

\subsection{Problem of section 7 in Landau}
Four-velocity
\begin{equation}
	u^i = \dv{x^i}{s} = \gamma_v \qty(1, \frac{\vb{v}}{c}),
\end{equation}
where
\begin{gather*}
	\gamma_v \equiv \frac{1}{\sqrt{1-\frac{v^2}{c^2}}}, \\
	\dd{s} = \sqrt{1-\frac{v^2}{c^2}} = \frac{c \dd{t}}{\gamma_v}.
\end{gather*}
Note that
\begin{equation}
	\dot{\gamma_v} = \frac{\gamma_v^3}{c^2}\vb{v}\vdot \dot{\vb{v}},
\end{equation}
we can compute the four-acceleration
\begin{align*}
	w^i &= \dv{u^i}{s} = \frac{\gamma_v}{c} \dv{u^i}{t} \\
	&= \frac{\gamma_v}{c} \qty[\dot{\gamma_v}\qty(1,\frac{\vb{v}}{c}) + \gamma_v \qty(0,\frac{\vb{v}}{c})] \\
	&= \frac{\gamma_v}{c} \qty[\frac{\gamma_v^3}{c^2} \vb{v}\vdot \dot{\vb{v}}\qty(1,\frac{\vb{v}}{c}) + \gamma_v \qty(0,\frac{\vb{v}}{c})] \\
	&= \frac{\gamma_v^2}{c^2} \qty(\frac{\gamma_v^2}{c} \vb{v} \vdot \dot{\vb{v}}, \frac{\gamma_v^2}{c^2} (\vb{v} \vdot \dot{\vb{v}}) \vb{v} + \dot{\vb{v}}).
\end{align*}
Consider only one-dimensional motion along $x$, using the inertial reference frame where the particle is at rest ($v=0$, so $\gamma_v = 1$), i.e., the particle-comoving inertial frame, the four-velocity in this comoving frame is
\begin{equation}
	u^i = (1, \vb{0}),
\end{equation}
and the four-acceleration is
\begin{equation}
	w^i = (0, \frac{w}{c^2}, 0, 0),
\end{equation}
where we define the acceleration as
\begin{equation}
	w \equiv \dv{v}{t},
\end{equation}
and $w$ is a constant in unifromly accelerated motion. The scalar product of four-acceleration is
\begin{equation}
	w^i w_i = -\frac{w^2}{c^4}.
\end{equation}
Now look at the ``fixed" reference frame where $v \neq 0$, the four-accerlation has components
\begin{align*}
	w^i &= \frac{\gamma_v^2}{c^2} \qty(\frac{\gamma_v^2}{c} v \dot{v}, \frac{\gamma_v^2}{c^2} v^2 \dot{v} + \dot{v}, 0, 0) \\
	&= \frac{\gamma_v^4}{c^2} \dot{v} \qty(\frac{v}{c}, \qty(\frac{v^2}{c^2} + \gamma_v^{-2}),0,0) \\
	&= \frac{\gamma_v^4}{c^2} \dot{v} \qty(\frac{v}{c}, 1,0,0),
\end{align*}
and the scalar product
\begin{equation}
	w^i w_i = \frac{\gamma_v^8}{c^4}\dot{v}^2 \qty(\frac{v^2}{c^2} - 1) = -\frac{\gamma_v^6}{c^4}\dot{v}^2.
\end{equation}
Notice that scalar is invariant under Lorentz transformation, so we have
\begin{equation}
	w^i w_i = -\frac{\gamma_v^6}{c^4}\dot{v}^2 = - \frac{w^2}{c^4},
\end{equation}
i.e.
\begin{equation}
	\frac{\gamma_v^3}{c^2}\dot{v} = w,
\end{equation}
which is equivalent to
\begin{equation}
	\dv{t} \qty(\gamma_v v) = w.
\end{equation}
Hence we reach to Landau's solution.

\subsection{A vector identity of section 17 Landau}
Show the following vector identity
\begin{equation}
	\grad{\vb{a} \vdot \vb{b}} = (\vb{a} \vdot \grad) \vb{b} + (\vb{b} \vdot \grad) \vb{a} + \vb{a} \cross (\curl{\vb{b}}) + \vb{b} \cross (\curl{\vb{a}}).
\end{equation}
\begin{proof}
	\begin{align*}
		[\grad{\vb{a} \vdot \vb{b}}]_i &= \partial_i (a_j b_j) = a_j \partial_i b_j + b_j \partial_i a_j \\
		&= a_j (\partial_i b_j - \partial_j b_i + \partial_j b_i) + b_j (\partial_i a_j - \partial_j a_i + \partial_j a_i) \\
		&= a_j \epsilon_{ijk} [\curl{\vb{b}}]_k + a_j \partial_j b_i + b_j \epsilon_{ijk} [\curl{\vb{a}}]_k + b_j \partial_j a_i \\
		&= (\vb{a} \vdot \grad) b_i + (\vb{b} \vdot \grad) a_i + [\vb{a} \cross (\curl{\vb{b}})]_i + [\vb{b} \cross (\curl{\vb{a}})]_i.
	\end{align*}
\end{proof}

\subsection{Derivation of problem 3 of section 22 in Landau, gradient-$B$ drift}
The general idea when we discuss charged particle drift in magnetic field is as follows. We must have some external force perpendicular to the magnetic field, such that on half of the cyclotron circle, the charged particle is accelerated and moves in a path with larger radius, while on the other half of the cyclotron circle, the charged particle is slowing down and moves with a tighter arc. The combination of arcs produces a translation perpendicular to both $\vb{B}$ and the force field $\vb{f}$. Recall the result of $E \cross B$ drift, the drift velocity is
\begin{equation}
	\vb{v}_d = c \frac{\vb{E} \cross \vb{B}}{B^2}. \label{eq: 2.4.1}
\end{equation}
For a general external force field $\vb{f}$, if $\vb{f}$ has component parallel to $\vb{B}$, this is irrelavant to the drift velocity, but accelerate the particle in the direction of $\vb{B}$. Hence without losing generality, we single out the perpendicular component $\vb{f}_{\perp}$ and call it $\vb{f}$ in the following discussion. The external force $\vb{f}$ should have the same effect for drift velocity as the electric field $\vb{E}$, so we do the substitution in \eqref{eq: 2.4.1}
\begin{align}
	\vb{E} &\to \vb{f}/e, \\
	\vb{v}_d &= c \frac{\vb{f} \cross \vb{B}}{e B^2}, \label{eq: 2.4.3}
\end{align}
where $e$ is the charge of the particle.

Now we are well prepared to discuss the gradient-$B$ drift in non-uniform magnetic field. We write the equation of the trajectory in the form
\begin{equation}
	\vb{r} = \vb{R}(t) + \vb*{\zeta}(t),
\end{equation}
where $\vb{R}(t)$ is the position vector of the guiding center (a slowly varying function of the time), while $\vb*{\zeta}(t)$ is a rapidly oscillating quantity discribing the rotational motion about the guiding center. The basic idea to find the drift velocity is averaging the force $e/c \dot{\vb{r}} \cross \vb{B}$ acting on the particle over a period of the oscillatory (circular) motion. We expand the function $\vb{B}(\vb{r})$ around the guiding center in powers of $\zeta$
\begin{equation}
	\vb{B}(\vb{r}) \approx \vb{B}(\vb{R}) + (\vb*{\zeta}\vdot \grad) \vb{B}(\vb{R}),
\end{equation}
where the gradient of magnetic field is evaluated at the position of guiding center. On averaging the terms of first order in $\zeta$ vanish, while the second-degree terms give rise to an additional force
\begin{equation}
	\vb{f} = \frac{e}{c} \dot{\vb*{\zeta}} \cross [(\vb*{\zeta}\vdot \grad) \vb{B}].
\end{equation}
For a circular orbit
\begin{equation}
	\dot{\vb*{\zeta}} = \vb*{\zeta} \cross \vb*{\omega}\qc \zeta = \frac{v_{\perp}}{\omega},
\end{equation}
where $\vb*{\omega}$ is the cyclotron frequency
\begin{equation}
	\vb*{\omega} = \frac{e \vb{B}}{\gamma m c},
\end{equation}
and $v_{\perp}$ is the velocity perpendicular to $\vb{B}$, i.e., in its circular motion. The average values of products of components of the vector $\vb*{\zeta}$, which is rotating in a plane (the plane perpendicular to $\vb{B}$), are
\begin{equation}
	\ev{\zeta_i \zeta_j} = \frac{1}{2}\zeta^2 \delta_{ij}.
\end{equation}
As a result we find
\begin{equation}
	\vb{f} = -\frac{\gamma m v_{\perp}^2}{2B^2} (\vb{B} \cross \grad) \vb{B}. \label{eq: 2.4.2}
\end{equation}
\begin{proof}
	We use tensor notation to prove \eqref{eq: 2.4.2}.
	\begin{align*}
		\vb{f} &\propto \ev{(\vb*{\zeta} \cross \vb{B}) \cross [(\vb*{\zeta} \vdot \grad) \vb{B}]}, \\
		f_i &\propto \ev{\epsilon_{ijk} (\epsilon_{jlm}\zeta_l B_m) [\zeta_r (\partial_r B_k)]} \\
		&= \frac{1}{2} \zeta^2 \delta_{lr} \epsilon_{ijk} \epsilon_{jlm} B_m (\partial_r B_k) \\
		&= \frac{1}{2} \zeta^2 \epsilon_{ijk} \epsilon_{jlm} B_m (\partial_l B_k) \\
		&= -\frac{1}{2} \zeta^2  \epsilon_{ijk} (\vb{B} \cross \grad)_j B_k \\
		&= -\frac{1}{2} \zeta^2 [(\vb{B} \cross \grad) \cross \vb{B}]_i.
	\end{align*}
\end{proof}
Next, we need to relate \eqref{eq: 2.4.2} to the gradient of magnetic field $\grad{B}$. Because the gradient operator does not act on the first $\vb{B}$ of \eqref{eq: 2.4.2}, we can separate its magnitude and direction
\begin{equation}
	\vb{f} = -\frac{\gamma m v_{\perp}^2}{2B} (\vu{n} \cross \grad) \vb{B},
\end{equation}
where $\vu{n}$ is a unit vector along $\vb{B}$, which is just a constant vector. Use the well known vector identity (BAC minus CAB) and one of the Maxwell equation $\div{\vb{B}} = 0$, we have
\begin{equation}
	(\vu{n} \cross \grad) \vb{B} = \grad{(\vu{n} \vdot \vb{B})} - \vu{n} (\div{\vb{B}}) = \grad{B}.
\end{equation}
Hence the force becomes
\begin{equation}
	\vb{f} = -\frac{\gamma m v_{\perp}^2}{2B} \grad{B}.
\end{equation}
Substitute into the expression for drift velocity \eqref{eq: 2.4.3} we obtain the well-known result for gradient-$B$ drift velocity
\begin{equation}
	\vb{v}_d = -\frac{\gamma m v_{\perp}^2}{2B} \frac{\grad{B} \cross \vb{B}}{B^2} = - \frac{1}{2}a v_{\perp} \frac{\grad{B} \cross \vb{B}}{B^2},
\end{equation}
where we use the centripetal force produced by magnetic field to define the cyclotron radius $a$:
\begin{equation}
	\gamma m \frac{v_{\perp}^2}{a} = eB v_{\perp},
\end{equation}
or
\begin{equation}
	a = \frac{v_{\perp}}{\omega}.
\end{equation}
For regions of space in which there are no currents, $\curl{\vb{B}} = 0$, which implies the magnetic field is bended
\begin{equation}
	\frac{\grad{B}}{B} = -\frac{\vb{R}}{R^2}.
\end{equation}
The drift velocity can be rewritten as
\begin{equation}
	\vb{v}_d = \frac{v_{\perp}^2}{2\omega R} \qty(\frac{\vb{R} \cross \vb{B}}{RB}).
\end{equation}

Another type of field variation that causes a drifting of the particle's guiding center is curvature of the lines of force. A particle spiral around the field lines with a cyclotron radius $a$ and a velocity $v_{\perp}$, while moving with a uniform velocity $v_{\parallel}$ along the lines of force. Suppose the lines of force are curved with a local radius of curvature $R$ that is large compared to $a$. The first-order motion can be understood as follows. The particle tends to spiral around a field line, but the field line curves off to the side. As far as the motion of the guiding center is concerned, this is equivalent to a centrifugal acceleration of magnitude $v_{\parallel}^2/R$. This acceleration can be viewed as arising from an effective electric field
\begin{equation}
	\vb{E}_{\text{eff}} = \frac{\gamma m \vb{R}}{e R^2} v_{\parallel}^2,
\end{equation}
where $\vb{R}$ is measured from the center of the curvature. The combined effective electric field and the magnetic field cause a curvature drift velocity
\begin{equation}
	\vb{v}_d \approx c \frac{\gamma m}{e} v_{\parallel}^2 \frac{\vb{R} \cross \vb{B}}{R^2 B^2} = \frac{v_{\parallel}^2}{\omega R} \qty(\frac{\vb{R} \cross \vb{B}}{RB})
\end{equation}

Finally we can combine the gradient drift and the curvature drift if there is no currents in space
\begin{equation}
	\vb{v}_d = \frac{1}{\omega R} (v_{\parallel}^2 + \tfrac{1}{2}v_{\perp}^2)\qty(\frac{\vb{R} \cross \vb{B}}{RB}).
\end{equation}

\end{document}
