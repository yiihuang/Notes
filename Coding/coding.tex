\documentclass[10pt]{article}
\pdfoutput=1
%\usepackage{NotesTeX,lipsum}
\usepackage{NotesTeX,lipsum}
%\usepackage{showframe}

\title{\begin{center}{\Huge \textit{Notes}}\\{{\itshape Coding}}\end{center}}
\author{Yi Huang\footnote{\href{https://yiihuang.com/}{\textit{My Personal Website}}}}


\affiliation{
University of Minnesota
}

\emailAdd{yihphysics@gmail.com}

\begin{document}
	\maketitle
	\flushbottom
	\newpage
	\pagestyle{fancynotes}
	\part{Caprice}
	\section{Fall 2018}\label{sec:fall2018}
	\begin{margintable}\vspace{.8in}\footnotesize
		\begin{tabularx}{\marginparwidth}{|X}
		Section~\ref{sec:fall2018}. Fall 2018\\
		\end{tabularx}
	\end{margintable}

	\subsection{Remove the margins of \texttt{.eps} files}

	We first install a command line tool \texttt{epstool} using \texttt{Homebrew}
	\begin{verbatim}
		brew update
		brew install epstool
	\end{verbatim}
	Next we use the following command to remove the margins and generate a \texttt{.pdf} file
	\begin{verbatim}
		epstool --copy --bbox input_file.eps input_file_temp.eps
		epstopdf --hires --outfile=output.pdf input_file_temp.eps
		rm input_file_temp.eps
	\end{verbatim}
	Enjoy it.

	\subsection{Transform \texttt{.pdf} to \texttt{.eps}}
	\begin{verbatim}
		brew install xpdf
		echo `export PATH="/usr/local/opt/qt/bin:$PATH"' >> ~/.bash_profile
	\end{verbatim}
	then 
	\begin{verbatim}
		pdftops -eps input_file.pdf
	\end{verbatim}

	\subsection{Crop the \texttt{.pdf} file}
	TeXLive itself has a tool called \texttt{pdfcrop}.
	\begin{verbatim}
		pdfcrop --margins `<left> <top> <right> <bottom>' input_file.pdf
	\end{verbatim}
	The unit in \texttt{<position>} is bp (big point), where 72 bp is 1 inch. Positive number is to append margin, while negative number is to crop the margin.
	Remarks: just use margin 0 to get rid of all the margins.
	\begin{verbatim}
		pdfcrop --margins 0 input_file.pdf
	\end{verbatim}

	

	\subsection{Reset the username and password of \texttt{GitHub}}

	Reset the password in website. Then when we use \texttt{git} in terminal, a warning will be presented \texttt{Authentication failed for}. To avoid this trouble, we need to update the information in our local environment.
	\begin{verbatim}
		$  git config --global --replace-all user.email "your email address"
		$  git config --global --replace-all user.name "your user name"
	\end{verbatim}


\end{document}
